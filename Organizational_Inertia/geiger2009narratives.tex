\section{Geiger and Antonacopoulou (2009)}

The key research paper appears to be \cite{geiger2009narratives}. This paper has been cited 120 times since 2009 as of 8 July 2020. The authors are active researchers in organisational studies.\\

Antonocopoulou having published several papers which have been cited more than 200 times each. She has been publishing research in management since the 1990s.\\

Geiger is not as cited as much with only having several being cited at least 40 times. Only one of Geiger's papers was cited more than 200 times. The current paper is his second most cited paper. He has been publishing research in management since 2009.\\

\subsection{Key Result}
The authors argue that:\\
\begin{quote}
 These insights into the self-reinforcing nature of narrative constructions could potentially shed new light on our understanding of blind spots in organizational perception which are often seen as antecedents of organizational inertia. {\bf The existing literature defines organizational inertia as the inability to enact internal change in the face of significant external change} \ldots . Blind spots are usually attributed to selective (competitor) perception \ldots, to myopic managerial decisions \ldots  or to cognitive inertia of middle or top managers \ldots . Our focus on narrative and narrative constructions can potentially provide an additional perspective for our understanding of blind spots in organizations. As our analysis has shown, narratives are extremely powerful in shaping organizational dynamics and their self-amplifying process of constructing self-legitimizing truth claims can create blind spots. These blind spots are not the result of managerial myopia or simple selective attention but rather are a product of the self-referential nature of narrative constructions in organizations. {\bf Despite the existence of critical voices (from inside as well as outside) the organization is aware of, the self-legitimizing nature of the narrative construction creates a self-reinforcing feedback mechanism which implicitly confirms dominant assumptions.} The blindness is therefore not the outcome of a purposeful act of domination which silences the deviating voices, but rather results from the circumstance that the functioning mechanism of the narrative construction is systematically hidden. By forming a web of self-reinforcing narratives which are self-legitimizing they limit the organization’s ability to “see” new perspectives or ‘listen’ to other voices. {\bf Silencing or choosing to ignore these alternative perspectives would limit the capacity of organizations to engage in fundamental change. Our narrative analysis shows how organizations become wedded to routines that reinforce the existing organizational culture and the dominant ways of doing things, thus reinforce the status quo and limit the scope for organizational learning to support the process of organizational change \ldots}. These insights extend our current understanding of the role of frames of reference in organizational change \ldots by drawing attention to the way such frames of reference emerge and become self-reinforcing.\\
\begin{flushright}
{\bf Emphasis Mine}
\end{flushright}
\end{quote}

An organisation has a story that most members believe and repeat as to why the organisation exists and is successful. This dominant narrative is difficult to challenge because as the old saying goes: "You can't argue with success".\\

\subsection{Journal Quality}
This paper is published in "The Journal of Applied Behavioral Science" which is a peer-reviewed journal.\\



\subsection{Top Publishing Results for Geiger}
The Google Scholar search for Daniel Geiger shows the following papers with at least 40 citations:\\

\begin{itemize}
\item "Revisiting the Concept of Practice: Toward an Argumentative Understanding of Practicing" (2009) cited 229 times
\item "The Significance of Distinctiveness: A Proposal for Rethinking Organizational Knowledge" (2007) cited 97 times
\item "Narratives in knowledge sharing: challenging validity" (2012) cited 56 times.
\item "Unravelling the Motor of Patterning Work: Toward an Understanding of the Microlevel Dynamics of Standardization and Flexibility" (2016) cited 54 times.
\item "Repairing Trust in an Organization after Integrity Violations: The Ambivalence of Organizational Rule Adjustments" (2015) cited 55 times.
\item "Turner in the Tropics:The Frontier Concept Revisited" (2009) cited 40 times (PhD thesis).
\item "Ever-Changing Routines? Toward a Revised Understanding of Organizational Routines Between Rule-Following and Rule-Breaking" (2014) cited 41 times
\end{itemize}

\subsection{Top Publishing Results for Antonacopoulou}
The Google Scholar search for Elena Antonacopoulou shows the following papers with at least 200 citations:\\
\begin{itemize}
\item "The Relationship between Individual and Organizational Learning: New Evidence from Managerial Learning Practices" (2006) cited 396 times.
\item "Emotion, learning and organizational change: Towards an integration of psychoanalytic and other perspectives" (2001) cited 365 times.
\item "Absorptive Capacity: A Process Perspective" (2008) cited 282 times.
\item "Reframing Competency In Management Development" (1996) cited 238 times.
\item "Making the Business School More ‘Critical’: Reflexive Critique Based on Phronesis as a Foundation for Impact" (2010) cited 245 times.
\item "The Social Complexity of Organizational Learning: The Dynamics of Learning and Organizing" (2007) cited 222 times.
\end{itemize}


