\section{Analysis of Geiger and Antonacopoulou 2009}

Here I analyse the citations to \cite{geiger2009narratives}.\\

\subsection{Weick 2012}

\cite[pp.143-144]{weick2012organized}:
\begin{quote}
N\"{a}slund  and  Pemer  demonstrate  the  ways  in  which  a  dominant  story  influences sensemaking and organizing. Dominant stories (Geiger and Antonacopoulou, 2009) may be a source of inertia but, from the perspective of organization theory, order that persists is  not  necessarily  a  bad  thing.  And,  if  that  persistence  is  attributed  to  stable  concepts given meaning by a dominant story, then process as well as structure are the explanations that link. N\"{a}slund and Pemer use the phrase ‘the resilience of dominant stories’ (p. 91) to describe their persistence. That suggests a family resemblance to the oft-cited concept of ‘normalization.’ N\"{a}slund and Pemer observe that dominant stories ‘[fix] the associative connotations of some of the central concepts’ that are needed to label and make sense of ‘organizational events such as good leadership, employee, consultant or project’ (p. 105). It is the mobilization and deployment of these very same associative connotations that occurs in the organizational practice of ‘normalization.’ Prior to the explosion of space shuttle Challenger, ‘technical anomalies that deviated from design performance expecta-tion were not interpreted as warning signs but became acceptable, routine and taken-for-granted aspects of shuttle performance’ (Vaughan, 2005: 34).\\
\begin{flushright}
\end{flushright}
\end{quote}

\subsection{Garud et al 2014}

\cite[p.17]{garud2014entrepreneurial}:
\begin{quote}
So  far,  we  have  discussed  how  and  why  ventures  may  fail  to  live  up  to  the  expectations  set  for stakeholders, thereby losing cognitive and pragmatic legitimacy in their eyes. Despite the legitimacy gap that emerges, entrepreneurial firms may still continue pursuing earlier paths as documented by Geiger and Antonacopoulou (2009). These researchers found that the dominant stories adopted by entrepreneurs of a bio-manufacturing company prevented them from questioning prior assumptions even when the company encountered a crisis. In other words, stories can become self-reinforcing, thereby generating cognitive and behavioral constraints(e.g. Tripsas \& Gavetti, 2000), and an escalation of commitment to a failing course of action(Staw, Sandelands, \& Dutton, 1981).
\begin{flushright}
\end{flushright}
\end{quote}

\subsection{Cayla and Arnould 2013}

\cite{cayla2013ethnographic}:
\begin{quote}
Fluid circulation of ethnographic stories.\\
Consistent with previous literature on organizational stories (Geiger and Antonacopoulou 2009), we find that ethnographic stories seem to take on a life of their own and add to the polyphonic narrative   flux   in   firms.   Unlike   written   reports   and   other vehicles of paradigmatic knowledge transfer, they resonate with clients, as Pascal explains:\ldots
\end{quote}

\subsection{Hanson 2012}

\cite[pp.696--7]{hansen2012business} mentions \cite{geiger2009narratives} in a footnote:\\
\begin{quote}
{\bf The claim for the importance of narrative appears in organizational research, but only a few organization scholars are explicitly discussing the role of historical narratives}.[12] An important implication in this liter-ature  is  that  not  only  historians,  but  also  historical  actors,  and  collec-tive entities such as organizations and other communities, create order in,  and  make  sense  of,  the  real  world  and  the  past  by  telling  stories.  Narratives are basic instruments for ordering reality, assigning causal-ity, and constructing meaning.[13] Humans—whether modern historians or  the  people  they  study—make  sense  of  the  world  by  telling  stories,  and these stories have the potential to frame the way members of an or-ganization  or  citizens  of  a  nation  see  the  world.  This  characteristic  of  narratives is exactly what makes history such a powerful tool, or even a weapon.  Thus,  history  in  the  shape  of  historical  narratives  is  a  basic  part of any group’s culture and identity, be it national or organizational, such as a company.\\
\begin{flushright}
{\bf Emphasis Mine} \\
\end{flushright}

Footnotes:\\

\ldots \\
12  See,  for  instance,  Andrew  D.  Brown,  “A  Narrative  Approach  to  Collective  Identities,”  Journal of Management Studies 43, no. 4 (2006): 731–53. Daniel Geiger and Elena Antona-copoulou,  “Narratives  and  Organizational  Dynamics:  Exploring  Blind  Spots  and  Organiza-tional Inertia,” Journal of Applied Behavioral Science 45, no. 3 (2009): 411–36. Eero Vaara, “On  the  Discursive  Construction  of  Success/Failure  in  Narratives  of  Post-Merger  Integra-tion,” Organization  Studies  23,  no.  2  (2002):  211–48.  Andrew  D.  Brown,  Yiannis  Gabriel,  and Silvia Gherardi, “Storytelling and Change: An Unfolding Story,” Organization 16, no. 3 (2009):  323–33.  One  scholar  who  deals  explicitly  with  historical  narratives  is  Olof  Brun-ninge, “Using History in Organizations: How Managers Make Purposeful Reference to His-tory  in  Strategy  Processes,”  Journal  of  Organizational  Change  Management  22,  no.  1  (2009): 8–26.\\
\end{quote}

\subsection{Simsek et al 2015}

\cite{simsek2015s}:
\begin{quote}
The  intuitive  appeal  of  imprinting  has  facilitated  its  diffusion  throughout  fields  such  as  organizational theory (e.g., Marquis \& Huang, 2010), strategic management (e.g., Ferriani, Garnsey, \& Lorenzoni, 2012), entrepreneurship (e.g., Milanov \& Shepherd, 2013), interna-tional  business  (e.g.,  Dieleman,  2010),  and  {\bf organizational  behavior  (e.g.,  Geiger  \&  Antonacopoulou,  2009)}.  Yet,  the  growing  dispersion  of  imprinting  research  and  resulting  fragmentation  of  empirical  insights  have  complicated  our  understanding  of  the  nature,  sources, and mechanisms of imprinting as well as the contexts in which imprinting shapes behavior and outcomes of distinct entities. Even as Marquis and Tilcsik’s (2013) recent mul-tilevel review and framework rigorously organize and discuss the imprinting literature, two central issues remain unresolved. First, while Marquis and Tilcsik provide insights into the multiple sources and entities of imprinting, we lack a framework for generalizing theoretical constructs, statements, and relationships across levels of analysis, contexts, and disciplinary boundaries.  Second,  while  research  suggests  a  distinction  between  the  process  by  which  imprints are formed and the subsequent mechanisms by which they evolve, we lack system-atic  insights  into  how  the  literature  on  the  elements,  mechanisms,  and  manifestations  of  imprinting dynamics and outcomes has developed. \\
\\
Many of the studies discussed above are in the context of founding a new organization. However, some studies examine the organization as an imprinter. {\bf For example, studies have examined how new employees can be inducted and socialized into the grand narrative of an organization from the time of hire, mitigating their ability to consider alternative viewpoints and challenge the status quo (Geiger \& Antonacopoulou, 2009)}. Relatedly, some studies have implicitly drawn upon the biological metaphor of “filial imprinting” (defined as the way in which offspring learn the behaviors of its parent) to describe how organizations may imprint spin-offs. In a number of studies, Klepper and colleagues have investigated the influence of the  prior  knowledge  and  capabilities  endowed  to  spin-offs  by  their  parent  firms  (Klepper,  2002; Klepper \& Simons, 2000; Roberts, Klepper, \& Hayward, 2011).\\
\\
Amplification of imprints.Going beyond the processes of maintenance and persistence, some have examined conditions under which imprints actually amplify and become increas-ingly ingrained or “inscribed” within an organization (Koch, 2011). These authors commonly invoke  path-dependence  arguments  to  illustrate  how  properties  such  as  increasing  returns  (Noda \& Collis, 2001) and non-ergodic state spaces (i.e., imprint selection restricts poten-tial future states; see Vergne \& Durand, 2010) result in reduced choice sets and increasing incentives  to  rely  on  imprinted  traits  or  decision  processes  (Schreyögg  \&  Sydow,  2011).  Interestingly,  this  also  partly  accounts  for  the  continued  debate  as  to  what  constitutes  the  unique domain of imprinting as compared to path dependence and other history-dependent temporal  processes  (for  detailed  comparisons,  see  Sydow,  Schreyögg,  \&  Koch,  2009;  and  Vergne \& Durand, 2010). Yet, path dependence is not the only mechanism by which imprints can  amplify,  providing  differentiation  between  the  two  concepts.  Scholars  have  pointed  to  escalation of commitment (Harris \& Ogbonna, 1999), {\bf self-legitimizing narratives (Geiger \& Antonacopoulou, 2009)}, performance feedback (Boeker, 1989b; Grandori \& Prencipe, 2008; Koch,  2011),  and  organizational  learning  (Dimov  et  al.,  2012;  Dowell  \&  Swaminathan,  2006) as additional mechanisms for reinforcing imprints.\\
\begin{flushright}
{\bf Emphasis Mine}
\end{flushright}
\end{quote}

\subsection{Antonacopoulou and Sheaffer 2014}

Antonacopoulou cites her own paper in \cite{antonacopoulou2014learning}:\\

\begin{quote}
Financial prosperity has come to symbolize effective man-agement and organization practices (Simons, 1999). However, many a manager contended Drucker (1994) is not alerted by the possible implications of success. Over time, routines that led to success institutionalize and preserve the modes of sen-semaking and core practices thus, laying the foundations for inertia (Kelly \& Amburgey, 1991; Krantz, 1988). {\bf By retreat-ing into an old and recognized behavioral pattern that in the past  meant  successful,  organizations  fall  into  the  trap  of  seeking  to  replicate  success  (Geiger  \&  Antonacopoulou,  2009)}.  This  point  raises  a  significant  distinction  between  replication  and  repetition  that  affect  learning  significantly  (Antonacopoulou,  2007).  By  seeking  to  replicate  past  suc-cess,  organizations  and  individuals  escalate  their  commit-ment  to  a  particular  course  of  action  (Whyte,  Sakes,  \&  Sterling, 1998). By wedding themselves to specific courses of actions that have once worked well, this leads to the develop-ment of institutional routines that in fact limit frames of refer-ence  with  which  managers  operate.  In  cases  where  such  frames  of  references  are  linked  to  success,  this  intensifies  egocentric  tendencies  among  managers  who  present  them-selves and their leadership as key to success. Therefore, over-confidence and complacency prompt leaders to act intuitively and impulsively. They become arrogant and conformists con-currently, in seeking to secure ongoing successes (Carmeli \& Sheaffer, 2009), which impairs their ability to make decisions and  amplifies  their  crisis  proneness  (Mellahi  et  al.,  2001).  Thus,  success  narrows  leaders’  perspectives,  changes  their  attitudes, boosts overconfidence in the ways they act (Ranft \&  O’Neill,  2001),  numbs  their  alertness,  and  locks  their  behavior  into  precarious  patterns  (Sheaffer  et  al.,  1998).  Consequently, a break with the external reality occurs owing to  the  collective  myopia  toward  competitive  challenges.  Under  circumstances  of  continued  success,  crises  are  per-ceived as temporary or secondary thus overemphasizing the effectiveness  of  past  strategies  as  a  basis  of  restoring  future  success (Bar-Joseph \& Sheaffer, 1998; Kisfalvi, 2000).
\begin{flushright}
{\bf Emphasis Mine}
\end{flushright}
\end{quote}

\subsection{Foster et al 2017}
\cite{foster2017strategic} only lists \cite{geiger2009narratives} in the bibliography.

\subsection{Tritten and Schoeneborn 2017}

\cite{trittin2017diversity}:
\begin{quote}
Organizational  scholars who are  drawing  on  the  notion  of  polyphony  tend  to  argue  that organizations  should  be  studied  as  discursive  spaces  that  should allow different  voices  to  be expressed (e.g., Hazen,  1993; Kornberger et al.,  2006; Pless,  1998). Viewing organizations as polyphonic phenomena implies  that  multiple  voices  are  dynamically  combined,  rather  than  merged,  in dialogue (Hazen, 1993) and may challenge authoritative tendencies (Kornberger et al.   ,   2006).   {\bf Works  that  have  been  influenced  by  Bakhtin’s  initial  ideas  emphasize  that  centripetal  forces  (i.e.,  hegemonic  tendencies  towards  harmonization)  and  centrifugal  forces  (i.e.,  resisting  voices  that  tend  toward  heterogeneity)  can  coexist  in  organizational  settings  (e.g.,   Brown,   2006;   Geiger   and   Antonacopoulou,   2009).}   However,   organizations   are   presumed to strive towards communicative homogeneity as “dominant discourses seek to shut down diversity by limiting the possibilities of sensemaking to a privileged and limited range of  discursive  strategies”  (Carter  and  Clegg,  2003,  p.  296).  Some scholars argue  that  17 
facilitating  polyphony  holds  the  potential  for  organizational  change (e.g.,  Carter  and  Clegg,  2003; Christensen et al., 2015; Jabri et al., 2008; Kornberger et al., 2006). Accordingly, these authors  suggest  following  the  ideal  of  the polyphonic  dialogue,   i.e.  a  moment  in  which  various dissonant voices are linked dialogically but without merging, as a guiding principle of organizing (e.g., Kornberger et al., 2006; for a critical view see Letiche, 2010). 
\begin{flushright}
{\bf Emphasis Mine}
\end{flushright}
\end{quote}

