\section{Overview}

These rankings were obtained from Google Scholar on 14 August 2021. \\

\section{Summary of Journals}

\begin{longtable}{|>{\raggedright\arraybackslash}p{3.7cm}|l|c|r|r|}
	
	\hline 
    {\bf Journal} & {\bf Peer Reviewed?} & {\bf h5-index} & {\bf h5-median}  \\
    \hline
    \endfirsthead
    \multicolumn{4}{c}%
    {\tablename\ \thetable\ -- \textit{Continued from previous page}} \\
    \hline
    {\bf Journal} & {\bf Peer Reviewed?} & {\bf h5-index} & {\bf h5-median}  \\
    \hline
    \endhead
    \hline \multicolumn{4}{r}{\textit{Continued on next page}} \\
    \endfoot
    \hline
    \endlastfoot

	Journal of business research & Yes & 140 & 199 \\
	\hline
	Sustainability & Yes & 103 & 143 \\
	\hline
	International Journal of Production Economics & Yes & 101 & 140 \\
	\hline
	Strategic Management Journal & Yes & 96 & 140 \\
	\hline
	MIS quarterly & Yes & 72 & 106 \\
	\hline
	Journal of Knowledge Management & Yes & 68 & 99 \\
	\hline
	Journal of the academy of marketing science & No & 64 & 107 \\
	\hline
	Journal of Management studies & No & 58 & 97 \\
	\hline
	Academy of Management Proceedings & No & 52 & 82 \\
	\hline
	Information Systems Research & Yes & 50 & 78 \\
	\hline
	Journal of Business \& Industrial Marketing & Yes & 46 & 59 \\
	\hline
	Journal of Organizational Change Management & Yes & 33 & 50 \\
	\hline
	Electronic commerce research & Yes & 28 & 36 \\
	\hline
	Management and Organization Review & Yes & 24 & 35 \\
	\hline
	Journal of technology management \& innovation & Yes & 20 & 26 \\
	\hline
	Korean Journal of Sociology & Yes & 7 & 11 \\
	\hline
	Journal of International Business Research & No & NA & NA \\
	\hline
	Formulating Research Methods for Information Systems & No & NA & NA \\
	\hline

    \caption{Journal Quality}
\end{longtable}

\subsection{Notes on Table of Journal Quality}

{\bf Peer Reviewed?} indicates that the submission of articles undergoes peer review according to the publication page. \\

{\bf h5-index} is the h-index for articles published in the last 5 complete years. It is the largest number h such that h articles published in 2015-2019 have at least h citations each. \\

{\bf h5-median} for a publication is the median number of citations for the articles that make up its h5-index. \\

{\bf NA} indicates that the journal did not appear in the list of top publications from Google Scholar.\\

Note that there are three (3) journals that do not appear to be peer-reviewed. \\

