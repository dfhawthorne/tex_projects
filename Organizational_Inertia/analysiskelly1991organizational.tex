\section{Analysis of Kelly and Amburgey 1991}

Here I analyse the citations to \cite{kelly1991organizational}.\\

\subsection{Aldrich 1999}

The PDF provided was a scan of chapter 2 of the book, "Organizations Evolving" (\cite{aldrich1999organizations}). \\

I could not find any reference to \cite{kelly1991organizational}.\\

\subsection{Gold et al 2001}

Gold et al \cite{gold2001knowledge} refers to Kelly and Amburgey's paper as reference \#67 in the paper. The paper is not searchable. \\

Used as source of description for preconditions of knowledge management as capabilities and resources \cite[p.186]{gold2001knowledge}.\\

Another reference to capability and resource in organizational behaviour literature \cite[p.192]{gold2001knowledge}.\\

\textit{In particular, the organization should experience a learning effect in which it improves over time in its capabilities for creating value.} \cite[p.196]{gold2001knowledge}.\\

The results of \cite{kelly1991organizational} are neither challenged or confirmed in this paper by Gold et al. \\

\subsection{Baum and Oliver 1991}

Baum and Oliver 1991 \cite{baum1991institutional} ("Institutional linkages and organizational mortality").\\

\cite[p.194]{baum1991institutional}:\\
\begin{quote}
Research on the effects of transformation on the risk of organizational failure has provided inconclusive results. Some studies report that transformation reduces the risk of failure (Zucker, 1987b). some report increases in the risk (Carroll, 1984; Miner, Amburgey, and Stearns, 1990), \textbf{some report no effect (Kelly and Amburgey, 1991)}, and still others report outcomes that depend on the type of change (Singh, House, and Tucker, 1986; Baum, 1990; Haveman, 1990) or amount of time since change (Amburgey, Kelly, and Barnett, 1990; Baum, 1990). One possible explanation for this inconsistency is that interorganizational linkages are capable of providing organizations with a "transformational shield" that "insulates an organization against the probability of failure resulting from transformation" (Miner, Amburgey, and Stearns, 1990: 695).
\begin{flushright}
\textbf{Emphasis Mine}
\end{flushright}
\end{quote}

\subsection{Fiss 2011}

Fiss 2011 \cite{fiss2011building} ("Building better causal theories: A fuzzy set approach to typologies in organization research").\\

\cite[p.397]{fiss2011building}:\\
\begin{quote}
Perhaps the most influential view of core versusperiphery  in  organizations  is  that  of  Hannan  and Freeman  (1984),  who  defined  an  organization’s core as its mission, authority structure, technology, and marketing strategy. In their definition, “Coreness  means  connectedness”  (Hannan  et  al.,  1996:506),  with  change  in  core  elements  requiring  adjustments in most other features of an organization. \textbf{This definition of core versus peripheral elements has been adopted in a considerable number of subsequent  studies  (e.g.,  Kelly  \&  Amburgey,  1991;}Singh, House, \& Tucker, 1986).
\begin{flushright}
\textbf{Emphasis Mine}
\end{flushright}
\end{quote}

\subsection{Lavie and Rosenkopf 2006}

Lavie and Rosenkopf 2006 \cite{lavie2006balancing} ("Balancing exploration and exploitation in alliance formation").\\

\cite[p.B2]{lavie2006balancing}:
\begin{quote}
\textbf{The Implications of Firm Age}.  The inertia perspective suggests that as firms mature, they are less likely to engage in exploration. \textbf{Older firms develop managerial commitment to existing technologies (Burgelman, 1994; Kelly and Amburgey, 1991), and are less likely to explore new technologies through their alliances.} In support of this argument, Rothaermel (2001b) found that incumbents benefited by exploiting complementary assets rather than by exploring new technologies with partners. In addition, firms nurture relationships as they mature and become embedded in alliance networks. This embeddedness encourages repeated alliances with prior partners (Gulati, 1995b; Li and Rowley, 2002), which are instituted on familiarity, trust, and established collaboration routines. Finally, maturation leads to the development of organizational routines that become embedded in decision-making processes and are applied almost automatically in response to external stimuli (Nelson and Winter, 1982). When a new problem arises, firms engage in local search for relevant experiences (Cyert and March, 1963) and elicit a response that conforms to their established routines.
\begin{flushright}
\textbf{Emphasis Mine}
\end{flushright}
\end{quote}

\cite[p.B6]{lavie2006balancing}:
\begin{quote}
With respect to firm age and partnering experience, our results clearly supported the slack and absorptive capacity perspectives with no contingencies across dimensions of exploration.  Older firms with substantial partnering experience tended to engage in exploration rather than exploitation in alliance formation. Hence, a firm’s accumulated experience is essential for expanding its knowledge domain and pursuing emerging opportunities that contribute to innovation and variation through alliance formation. While experience may also lead to the evolution of rigid routines and inertial forces, we found no evidence that such processes enforce exploitation. It is possible, however, that inertial forces reduce the efficiency of exploration without constraining the tendency to explore, at least in the context of alliances. \textbf{While our findings are consistent with prior research showing increase in the rate of innovation with firm age (Sorensen and Stuart, 2000), they stand in contrast with the inertia literature that argued for limited responsiveness of older firms to organizational change (Amburgey, Kelly, and Barnett, 1993; Hannan and Freeman, 1984; Kelly and Amburgey, 1991).} Possibly, older firms that encounter inertial impediments for innovation explore through alliances while pursuing exploitation through their internal units.
\begin{flushright}
\textbf{Emphasis Mine}
\end{flushright}
\end{quote}

