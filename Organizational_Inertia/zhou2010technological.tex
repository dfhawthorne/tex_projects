\section{Zhou 2010}

\subsection{Summary}

Although \cite{zhou2010technological} uses \textit{organizational inertia theory} as an integral part, the article is not critiqued on \textit{organizational inertia theory} in the few articles reviewed from freely-available sources.

\subsection{Article Statistics}

\cite{zhou2010technological} is cited 1,150 times as of 14 August 2021. It is published in \textit{Strategic Management Journal} which is not peer-reviewed, but has a h5-index of 96.

\subsection{abstract}

Only the abstract was available:

\begin{quotation}
	This paper examines the role of technological capability in product innovation. Building on the absorptive capacity perspective and \textbf{organizational inertia theory}, the authors propose that technological capability has curvilinear and differential effects on exploitative and explorative innovations. The findings support the proposition that though technological capability fosters exploitation at an accelerating rate, it has an inverted U-shaped relationship with exploration. That is, a high level of technological capability impedes explorative innovation. Strategic flexibility strengthens the positive effects of technological capability on exploration, such that when strategic flexibility is high, greater technological capability is associated with more explorative innovation.
\end{quotation}

Organizational inertia theory is an integral part of this paper.

\subsection{Summary of Articles Citing Zhou 2010}

The articles listed below all have freely available PDFs for review from the first ten (10) results returned by Google Scholar on 17 August 2021.

\begin{longtable}{|>{\raggedright\arraybackslash}p{3.7cm}|>{\raggedright\arraybackslash}p{6cm}|>{\arraybackslash}p{1.5cm}|}
	
	\hline 
    {\bf Reference} & {\bf Title} & {\bf \# cited}  \\
    \hline
    \endfirsthead
    \multicolumn{3}{c}%
    {\tablename\ \thetable\ -- \textit{Continued from previous page}} \\
    \hline
    {\bf Reference} & {\bf Title} & {\bf Number of times cited}  \\
    \hline
    \endhead
    \hline \multicolumn{3}{r}{\textit{Continued on next page}} \\
    \endfoot
    \hline
    \endlastfoot

	\cite{casadesus2013business} & Business model innovation and competitive imitation: The case of sponsor-based business models & 814 \\
	\hline
	\cite{haans2016thinking} & Thinking about U: Theorizing and testing U-and inverted U-shaped relationships in strategy research & 708 \\
	\hline
	\cite{lin2014exploring} & Exploring the role of dynamic capabilities in firm performance under the resource-based view framework & 700 \\
	\hline
	\cite{bock2012effects} & The effects of culture and structure on strategic flexibility during business model innovation & 519 \\
	\hline 
	\cite{xu2013linking} & Linking theory and context:‘Strategy research in emerging economies’ after Wright et al.(2005) & 425 \\
	\hline
	\cite{cui2016utilizing} & Utilizing customer knowledge in innovation: antecedents and impact of customer involvement on new product performance & 384 \\
	\hline
	\cite{fores2016does} & Does incremental and radical innovation performance depend on different types of knowledge accumulation capabilities and organizational size? & 374 \\
	\hline
	\cite{zawislak2012innovation} & Innovation capability: From technology development to transaction capability & 330 \\
	\hline

    \caption{Articles that cite Zhou 2010 (with PDFs)}
\end{longtable}

\subsection{Review of Citations to Zhou 2010}

\subsubsection{Casadesus 2013}

\cite{casadesus2013business} \\

\subsubsection{Haan 2016}

\cite{haans2016thinking} has no reference to inertia. \\

\subsubsection{Lin 2014}

\cite{lin2014exploring} has no reference to inertia. \\

\subsubsection{Bock 2012}

\cite{bock2012effects} has only one (1) reference to \textit{inertia}, but it is not from \cite{zhou2010technological}: \\
\begin{quotation}
\ldots In contrast,
business model innovation could also require unlearning partner-specific routines which
could act as inertial impediments to flexibility. \ldots
\end{quotation}

\subsubsection{Xu 2013}

\cite{xu2013linking} has no reference to \textit{inertia}. \\

\subsubsection{Cui 2016}

\cite{cui2016utilizing} has one (1) reference to \textit{inertia}, but it is not from \cite{zhou2010technological}: \\
\begin{quotation}
Firm age is also included because older firms tend to be
less innovative due to organizational inertia (Phelps 2010).
\end{quotation}

\subsubsection{Fores 2016}

\cite{fores2016does} has two (2) references to \textit{inertia}.\\

The first is from \cite{zhou2010technological}:\\
\begin{quotation}
\ldots 
Moreover, the pressure of organizational inertia intensifies as a firm accumulates
extensive technological and other organizational skills and forms its unique processes and
routines \textbf{(Zhou and Li, 2010)}. \ldots
\end{quotation}

The other is not from \cite{zhou2010technological}:\\
\begin{quotation}
\ldots Larger firms are likely to have greater resources and capabilities that allow them
to extend their existing knowledge base. That is, larger firms devote more effort to
accumulating knowledge that perpetuates the innovation performance arising from their
consolidated research lines. This path dependence in large corporations creates bureaucratic
and cultural sources of structural inertia that can inhibit the entrepreneurial spirit of their
employees to introduce radical innovation performance. \ldots
\end{quotation}

\subsubsection{Zawislak 2012}

\cite{zawislak2012innovation} has no mention of \textit{organizational inertia}:

