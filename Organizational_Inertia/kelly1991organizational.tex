\section{Kelly and Amburgey 1991}

\subsection{Overview}

This paper is called \textit{Organizational inertia and momentum: A dynamic model of strategic change} (\cite{kelly1991organizational}).\\

\subsection{Summary of Citations of this Article}

The Google Scholar search for "organizational inertia" on 10 July 2020 returned the following top  ten (10) results:\\
\begin{longtable}{
    |>{\raggedright\arraybackslash}p{3.7cm}
    |>{\raggedright\arraybackslash}p{3.7cm}
    |r
    |r
    |c
    |}
	
	\hline 
    \textbf{Reference} & \textbf{Title} & \textbf{Published} & \textbf{Cited}  & \textbf{PDF} \\
    \hline
    \endfirsthead
    \multicolumn{5}{c}%
    {\tablename\ \thetable\ -- \textit{Continued from previous page}} \\
    \hline
    \textbf{Reference} & \textbf{Title} & \textbf{Published} & \textbf{Cited}  & \textbf{PDF} \\
    \hline
    \endhead
    \hline \multicolumn{5}{r}{\textit{Continued on next page}} \\
    \endfoot
    \hline
    \endlastfoot

\cite{aldrich1999organizations} & Organizations evolving & 1999 & 6,915 & Yes \\
\hline
\cite{shane2003general} & A general theory of entrepreneurship: The individual-opportunity nexus & 2003 & 5,676 & No \\
\hline
\cite{gold2001knowledge} & Knowledge management: An organizational capabilities perspective & 2001 & 5,261 & Yes \\
\hline
\cite{tolbert2015organizations} & Organizations: Structures, processes and outcomes & 2015 & 2,688 & No \\
\hline
\cite{armenakis1999organizational} & Organizational change: A review of theory and research in the 1990s & 1999 & 2,249 & No \\
\hline
\cite{baum1991institutional} & Institutional linkages and organizational mortality & 1991 & 2,196 & Yes \\
\hline
\cite{ruekert1992developing} & Developing a market orientation: an organizational strategy perspective & 1992 & 1,820 & No \\
\hline
\cite{langlois1995firms} & Firms, markets and economic change: A dynamic theory of business institutions & 1995 & 1,732 & No \\
\hline
\cite{fiss2011building} & Building better causal theories: A fuzzy set approach to typologies in organization research & 2011 & 1,737 & Yes \\
\hline
\cite{lavie2006balancing} & Balancing exploration and exploitation in alliance formation & 2006 & 1,597 & Yes \\
\hline
\caption{Citations of Kelly and Amburgey 1991}
\end{longtable}

What is interesting is that four (4) works that cite this article are books (\cite{aldrich1999organizations}, \cite{shane2003general}, \cite{tolbert2015organizations}, and \cite{langlois1995firms}). \\

There are five (5) freely available PDFs available for me to examine to see how this article is cited. \\

 All of the top ten (10) references have themselves been cited over 1,000 times each. This area of management research does not seem to be peripheral. \\
 
\subsection{Key Results}

\cite[p. 594]{kelly1991organizational} says:
\begin{quote}
Hannan and  Freeman  (1984:156)  contended  that constraints  on  change in  the  core  features  of  organizations  are  very  strong.  In  comparison  to  the probability  of change in peripheral organizational features, the probability  of change  in  core features  is low  (Hannan  \& Freeman,  1984: 157). Hannan  and Freeman  did  not  suggest  that  organizations  never  change. Instead,  \textbf{they  defined  inertia  relative  to environmental  change: "Structures  of  organizations have high  inertia  when  the  speed  of  reorganization  [core  feature  change]  is much lower than the rate at which environmental conditions change" (1984:151).}
\begin{flushright}
\textbf{Emphasis Mine}
\end{flushright}
\end{quote}
 
\cite[pp. 608--9]{kelly1991organizational} says:
\begin{quote}
 There  was  no  support  for  the  inertia  theory  prediction  that  organizational  size  is associated  with  a decrease  in the probability  of change,  unless we include  prior experience with change in the model. Even then, we  found support  for  only  one  of  the  four  changes  in  strategic  orientation:  a  large airline  is  less  likely  to  change  to  business-level  generalism  than  a  small airline.  Hannan  and  Freeman  (1984:  158-162)  discussed  how  complicated the relationship between organizational  size and core feature change may be. Large  organizations  face  not  only  low  flexibility  because  of  bureaucratization  and  formalization  but  also  long  durations  of  attempts  to  bring  about change.  Without  data  on  organizational  complexity  and  the  duration  ofchange  attempts,  it was  impossible  to  assess  that  part  of their  argument.  At best,  we  found  a  weak  relationship  between  organizational  size  and  the probability  of  change  in this  population.  This finding  should  be  interpreted with  caution,  however,  because  our  measure  of  size  is  total  airline  assets rather than total organizational  assets. Also, we lacked  size data on 42  of the 178 airlines, and  the  size  variable  was  left-censored.\\
 
 We  definitely   saw  momentum  in  organizational  change  processes. These  organizations  were  significantly  more  likely  to  repeat  changes  that they had  experienced  in the past. \textbf{We suggest that the concept  of  momentum is  complementary  to  inertia  theory  and  that  a  useful  way  to  think  about inertia  is  that  it  is  high  when  organizations  continue  to  extrapolate  past trends  in the face  of environmental  change.} But before  we can conclude  that our findings  are indicators  of organizational "tracks" (Hinings \& Greenwood,1988) or archetypes  (Miller \& Friesen,  1984), we will need additional  data on such internal  organizational  characteristics  as culture, power, decision  making, communication,  leadership, and the like. This is an important  issue that warrants  further  investigation.\\
 
This  study  expanded  Hannan  and  Freeman's  theory  by  demonstrating the  utility  of  including  cumulative  prior  organizational  changes  to  capture momentum  in  organizational  change  processes.  In  our  view,  studies  of  organizational  change  require  historical  perspective.  What  can  appear  in  isolation to be a discontinuous  change in strategic orientation  can actually  be a manifestation  of  momentum.
\begin{flushright}
\textbf{Emphasis Mine}
\end{flushright}
\end{quote}

The conclusion (\cite[p. 610]{kelly1991organizational}), in part, says:
\begin{quote}
Our research also suggests that although managers have more  discretion regarding  cbange  in  their  organizations  than  ecological  theorists  have  typically proposed,  managers remain  constrained  by organizational  history.  Old airlines  are  more  resistant  to  change  than  young  ones.  More  important, prior  organizational  actions  bave  a  powerful  effect  on  both  the  probability and  content  of  strategic  change.  At  the  level  of  both  business  units  and  an entire corporation, history constrains the choices available to managers. This constraint  has  an  important  implication  for  managers  who  desire  organizational  change: \textbf{a true  shift  in orientation  requires  historical  perspective,  and a  series  of  incremental  changes  may  be  more  feasible  and  effective  than  a single  major  shift  in  a particular  strategic  direction.} \\
\begin{flushright}
\textbf{Emphasis Mine}
\end{flushright}
\end{quote}

