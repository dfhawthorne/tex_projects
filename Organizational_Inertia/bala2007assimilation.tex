\section{Bala and Venkatesh 2007}

\subsection{Summary}


\subsection{Abstract}

The abstract of \cite{bala2007assimilation} is:\\
\begin{quotation}
Organizations have not fully realized the benefits of interorganizational relationships (IORs) due to the lack of cross-enterprise process integration capabilities. Recently, interorganizational business process standards (IBPS) enabled by information technology (IT) have been suggested as a solution to help organizations overcome this problem. Drawing on three theoretical perspectives, i.e., the relational view of the firm, institutional theory, and \textbf{organizational inertia theory}, we propose three mechanisms—relational, influence, and inertial—to explain the assimilation of IBPS in organizations. We theorize that these mechanisms will have differential effects on the assimilation of IBPS in dominant and nondominant firms. Using a cross-case analysis based on data from 11 firms in the high-tech industry, we found evidence to support our propositions that relational depth, relationship extendability, and normative pressure were important for dominant firms while relational specificity and influence mechanisms (coercive, mimetic, and normative pressures) were important for nondominant firms. Inertial mechanisms, i.e., ability and willingness to overcome resource and routine rigidities, were important for both dominant and nondominant firms.
\end{quotation}

\subsection{Articles Citing Bala and Venkatesh 2007}

\begin{longtable}{
		|>{\raggedright\arraybackslash}p{3.7cm}
		|>{\raggedright\arraybackslash}p{6cm}
		|r
		|}
	
	\hline 
	{\bf Reference} & {\bf Title} & {\bf No. times cited}  \\
	\hline
	\endfirsthead
	\multicolumn{3}{c}%
	{\tablename\ \thetable\ -- \textit{Continued from previous page}} \\
	\hline
	{\bf Reference} & {\bf Title} & {\bf No. times cited}  \\
	\hline
	\endhead
	\hline \multicolumn{3}{r}{\textit{Continued on next page}} \\
	\endfoot
	\hline
	\endlastfoot
	
	\cite{venkatesh2013bridging} & Bridging the qualitative-quantitative divide: Guidelines for conducting mixed methods research in information systems & 2843 \\
	\hline
	\cite{rai2010leveraging} & Leveraging IT capabilities and competitive process capabilities for the management of interorganizational relationship portfolios & 368 \\
	\hline
	\cite{rai2012interfirm} & Interfirm IT capability profiles and communications for cocreating relational value: evidence from the logistics industry & 352 \\
	\hline
	\cite{mignerat2015positioning} & Positioning the institutional perspective in information systems research & 342 \\
	\hline
	\cite{sila2013factors} & Factors affecting the adoption of B2B e-commerce technologies & 271 \\
	\hline
	\cite{zhang2009investigation} & An investigation of resource-based and institutional theoretic factors in technology adoption for operations and supply chain management & 235 \\
	\hline
	\cite{venkatesh2012adoption} & Adoption and impacts of interorganizational business process standards: Role of partnering synergy & 217 \\
	\hline

\caption{Articles that cite Bala and Venkatesh 2007 (with PDFs)}
\end{longtable}


