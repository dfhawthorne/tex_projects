\subsection{Overview}

I choose the top thirty (30) search results from {\tt https://scholar.google.com.au} . I evaluated each result through the number of times and how the article was cited.

\subsection{Results}

The Google Scholar search for "organizational inertia theory" on 14 August 2021 returned the following top 30 results:\\
\begin{longtable}{
    |>{\raggedright\arraybackslash}p{3.7cm}
    |>{\raggedright\arraybackslash}p{3.7cm}
    |r
    |r
    |c
    |}
	
	\hline 
    \textbf{Reference} & \textbf{Title} & \textbf{Published} & \textbf{Cited}  & \textbf{PDF} \\
    \hline
    \endfirsthead
    \multicolumn{5}{c}%
    {\tablename\ \thetable\ -- \textit{Continued from previous page}} \\
    \hline
    \textbf{Reference} & \textbf{Title} & \textbf{Published} & \textbf{Cited}  & \textbf{PDF} \\
    \hline
    \endhead
    \hline \multicolumn{5}{r}{\textit{Continued on next page}} \\
    \endfoot
    \hline
    \endlastfoot

	\cite{zhen2021impact} & Impact of organizational inertia on organizational agility: the role of IT ambidexterity & 2021 & 0 & No \\
	\hline 
	\cite{haag2014organizational} & Organizational inertia as barrier to firms' IT adoption–multidimensional scale development and validation & 2014 & 36 & Yes \\
	\hline 
	\cite{nedzinskas2014dynamic} & Dynamic capabilities and organizational inertia interaction in volatile environment & 2014 & 14 & Yes \\
	\hline 
	\cite{jui2020study} & The Study of Organizational Inertia, Business Model Innovation and Organizational Performance in Taiwan Financial Institutions: Organizational Learning Perspective & 2020 & 0 & Yes \\
	\hline 
	\cite{qian2014chapter} & CHAPTER THIRTEEN THE ROLES OF STRATEGICALLIANCES IN DYNAMIC ENVIRONMENTS: AN ORGANIZATIONAL INERTIA PERSPECTIVE & 2014 & 0 & No \\
	\hline 
	\cite{wei2011bottom} & Bottom‐up learning, organizational formalization, and ambidextrous innovation & 2011 & 65 & Yes \\
	\hline 
	\cite{airikkala2021dynamic} & Dynamic capabilities and organizational inertia during digital transformation & 2021 & 0 & Yes \\
	\hline 
	\cite{liang2017unraveling} & Unraveling the alignment paradox: how does business—IT alignment shape organizational agility? & 2017 & 75 & No \\
	\hline 
	\cite{zhou2010technological} & Technological capability, strategic flexibility, and product innovation & 2010 & 1,150 & No \\
	\hline 
	\cite{liu2013review} & A Review Of Organizational Ecology Environment Selection Mechanism Research & 2013 & 0 & No \\
	\hline 
	\cite{kelly1989test} & A test of organizational inertia theory & 1989 & 6 & No \\
	\hline
	\cite{yusof2020exploring} & Exploring the Elements of Organizational Inertia and Impacts on Organization. & 2020 & 0 & No \\
	\hline
	\cite{mezzour2010three} & Three essays on internationalization and firm entrepreneurial behavior & 2010 & 0 & No \\
	\hline
	\cite{huang2021does} & When does servitization promote product innovation? & 2021 & 0 & No \\
	\hline
	\cite{cheng2013breakthrough} & Breakthrough innovation: the roles of dynamic innovation capabilities and open innovation activities & 2013 & 140 & No \\
	\hline
	\cite{guiwen2019incumbent} & Incumbent Enterprises' Resource and Business Model Innovation in the Context of Technological Change: The Mediating Role of Binary Dynamic Capabilities & 2019 & 0 & No \\
	\hline
	\cite{bala2007assimilation} & Assimilation of interorganizational business process standards & 2017 & 271 & No \\
	\hline
	\cite{yi2018relationship} & Relationship between Slack Resources and New Product Development Performance under the Internet Environment & 2018 & 0 & No \\
	\hline
	\cite{polthierseeing} & Seeing the Good in the Bad–Leveraging Customer Complaints for New Product Development & ? & 0 & Yes \\
	\hline
	\cite{yu2020organizational} & Organizational search and business model innovation: the moderating role of knowledge inertia & 2020 & 3 & No \\
	\hline
	\cite{popli2020value} & Value Constraining or Value Enabling? The Impact of Business Group Affiliation on Post-Acquisition Performance by Emerging Market Firms & 2020 & 4 & No \\
	\hline
	\cite{yi2017bottom} & Bottom-up learning, strategic flexibility and strategic change & 2017 & 32 & Yes \\
	\hline
	\cite{hao2019big} & Big data, big data analytics capability, and sustainable innovation performance & 2019 & 7 & No \\
	\hline
	\cite{bruggeman1997formalizing} & Formalizing organizational inertia theory: A didactic essay & 1997 & 0 & No \\
	\hline
	\cite{jung2009environmental} & Environmental Turbulence and Organizational Change: Niche Expansion and Organizational Growth in the Population of Hospitals in Korea, 1980-2008 & 2009 & 2 & Yes \\ 
	\hline
	\cite{gligor2021theorizing} & Theorizing the dark side of business-to-business relationships in the era of AI, big data, and blockchain & 2021 & 0 & No \\
	\hline
	\cite{baum1990inertial} & INERTIAL AND ADAPTIVE PATTERNS IN ORGANIZATIONAL CHANGE. & 1990 & 77 & Yes \\
	\hline
	\cite{hughes1998perspectives} & Perspectives on policing: a synopsis of recent research & 1998 & 0 & No \\
	\hline
	\cite{ng1996micro} & Micro-economic evolution of the firm: an organizational ecology perspective & 1996 & 1 & Yes \\
	\hline 
	\cite{cortez2015innovation} & Innovation and financial performance of electronics companies: a cross-country comparison & 2015 & 12 & No \\
	\hline 	
	\caption{Google scholar search results for organizational inertia theory in descending order of relevance} 
\end{longtable} 



Of the 30 results selected, 11 have freely available PDF files for downloaded.\\

It is worrying that only 16 have been cited at all with only 3 more than 100 times. And only one has been cited more than 1,000 times.\\

\begin{longtable}{
		|>{\raggedright\arraybackslash}p{3.7cm}
		|>{\raggedright\arraybackslash}p{3.7cm}
		|r
		|r
		|c
		|}
	
	\hline 
	\textbf{Reference} & \textbf{Title} & \textbf{Published} & \textbf{Cited}  & \textbf{PDF} \\
	\hline
	\endfirsthead
	\multicolumn{5}{c}%
	{\tablename\ \thetable\ -- \textit{Continued from previous page}} \\
	\hline
	\textbf{Reference} & \textbf{Title} & \textbf{Published} & \textbf{Cited}  & \textbf{PDF} \\
	\hline
	\endhead
	\hline \multicolumn{5}{r}{\textit{Continued on next page}} \\
	\endfoot
	\hline
	\endlastfoot

	\cite{zhou2010technological} & Technological capability, strategic flexibility, and product innovation & 2010 & 1,150 & No \\
	\hline 
	\cite{bala2007assimilation} & Assimilation of interorganizational business process standards & 2017 & 271 & No \\
	\hline
	\cite{cheng2013breakthrough} & Breakthrough innovation: the roles of dynamic innovation capabilities and open innovation activities & 2013 & 140 & No \\
	\hline
	\cite{baum1990inertial} & INERTIAL AND ADAPTIVE PATTERNS IN ORGANIZATIONAL CHANGE. & 1990 & 77 & Yes \\
	\hline
	\cite{liang2017unraveling} & Unraveling the alignment paradox: how does business—IT alignment shape organizational agility? & 2017 & 75 & No \\
	\hline 
	\cite{wei2011bottom} & Bottom‐up learning, organizational formalization, and ambidextrous innovation & 2011 & 65 & Yes \\
	\hline 
	\cite{haag2014organizational} & Organizational inertia as barrier to firms' IT 	adoption–multidimensional scale development and validation & 2014 & 36 & Yes \\
	\hline 
	\cite{yi2017bottom} & Bottom-up learning, strategic flexibility and strategic change & 2017 & 32 & Yes \\
	\hline
	\cite{nedzinskas2014dynamic} & Dynamic capabilities and organizational inertia interaction in volatile environment & 2014 & 14 & Yes \\
	\hline 
	\cite{cortez2015innovation} & Innovation and financial performance of electronics companies: a cross-country comparison & 2015 & 12 & No \\
	\hline 	
	\cite{hao2019big} & Big data, big data analytics capability, and sustainable innovation performance & 2019 & 7 & No \\
	\hline
	\cite{kelly1989test} & A test of organizational inertia theory & 1989 & 6 & No \\
	\hline
	\cite{popli2020value} & Value Constraining or Value Enabling? The Impact of Business Group Affiliation on Post-Acquisition Performance by Emerging Market Firms & 2020 & 4 & No \\
	\hline
	\cite{yu2020organizational} & Organizational search and business model innovation: the moderating role of knowledge inertia & 2020 & 3 & No \\
	\hline
	\cite{jung2009environmental} & Environmental Turbulence and Organizational Change: Niche Expansion and Organizational Growth in the Population of Hospitals in Korea, 1980-2008 & 2009 & 2 & Yes \\ 
	\hline
	\cite{ng1996micro} & Micro-economic evolution of the firm: an organizational ecology perspective & 1996 & 1 & Yes \\
	\hline 

	\caption{Google scholar search results for organizational inertia theory in descending order of number of times cited} 
\end{longtable} 
