\documentclass[]{report}
\usepackage{array}
\usepackage{longtable}
\usepackage{natbib}
\usepackage{graphicx}
\usepackage{hyperref}


% Title Page
\title{Literature Review on Alarm Fatigue Evidence Based Practice}
\author{Douglas Hawthorne}
\bibliographystyle{dinat}

\begin{document}
	\maketitle
	
	\begin{abstract}
		
	\end{abstract}

	\tableofcontents
	\listoftables
	% \listoffigures
	
	\chapter{Introduction}
	I am interested in studying alarm fatigue, especially in relation to computer operations.\\
	
	This literature review is limited to articles published and catalogued through Google Scholar.\\
	
	\chapter{Search Results}
	
	\section{Overview}
	
	I used Google Scholar to return results based on what it considered to be of relevance. Only the first two (2) pages of results were reviewed. Deeper analysis was only done for papers freely available online.
	
	\section{Google Scholar Search Results}
	
	Based on the phrase, `"alarm fatigue" evidence based practice', I got the following results from Google Scholar on 14 March 2024. These result were sorted by relevance as determined by Google Scholar. Only results from the first two (2) pages were considered:\\
	\begin{longtable}{
			|>{\raggedright\arraybackslash}p{3.7cm}
			|>{\raggedright\arraybackslash}p{3.7cm}
			|r
			|r
			|c
			|}
		
		\hline 
		\textbf{Reference} & \textbf{Title} & \textbf{Published} & \textbf{Cited}  & \textbf{PDF} \\
		\hline
		\endfirsthead
		\multicolumn{5}{c}%
		{\tablename\ \thetable\ -- \textit{Continued from previous page}} \\
		\hline
		\textbf{Reference} & \textbf{Title} & \textbf{Published} & \textbf{Cited}  & \textbf{Available Free Online} \\
		\hline
		\endhead
		\hline \multicolumn{5}{r}{\textit{Continued on next page}} \\
		\endfoot
		\hline
		\endlastfoot
		
		\citet{cvach2012monitor} & \textit{Monitor Alarm Fatigue: An Integrative Review} & 2012 & 718 & Yes \\
		\hline
		\citet{rayo2016diagnosing} & \textit{Diagnosing and treating “alarm fatigue”: pragmatic and evidence-based approaches needed} & 2016 & 4 & Yes \\
		\hline
	    \citet{dee2022determining} & \textit{Determining the impact of an alarm management program on alarm fatigue among ICU and Telemetry RNs: An evidence based research project} & 2022 & 10 & Yes \\
        \hline
		\citet{gorisek2021evidence} & \textit{An evidence-based initiative to reduce alarm fatigue in a burn intensive care unit} & 2021 & 8 & No \\
        \hline
	    \citet{welch2011evidence} & \textit{An evidence-based approach to reduce nuisance alarms and alarm fatigue} & 2011 & 129 & Yes \\
        \hline
	    \citet{srinivasa2017evidence} & \textit{An Evidence‐Based Approach to Reducing Cardiac Telemetry Alarm Fatigue} & 2017 & 48 & No \\
        \hline
		\citet{sendelbach2013alarm} & \textit{Alarm fatigue: a patient safety concern} & 2013 & 489 & No \\
		\hline
		\citet{johnson2017alarm} & \textit{Alarm safety and alarm fatigue} & 2017 & 104 & No \\
	    \hline
		\citet{lewis2019research} & \textit{Research Outcomes of Implementing CEASE
	An Innovative, Nurse-Driven, Evidence-Based, Patient-Customized Monitoring Bundle to Decrease Alarm Fatigue in the Intensive Care Unit/Step-down Unit} & 2021 & 56 & No \\
        \hline
        \citet{oliveira2018alarm} & \textit{Alarm fatigue and the implications for patient safety} & 2018 & 37 & Yes \\
        \hline
        \citet{bi2020effects} & \textit{Effects of monitor alarm management training on nurses' alarm fatigue: A randomised controlled trial} & 2020 & 64 & No \\
        \hline
        \citet{casey2018critical} & \textit{Critical care nurses' knowledge of alarm fatigue and practices towards alarms: A multicentre study} & 2018 & 86 & Yes \\
        \hline
        \citet{scott2019mechanical} & \textit{Mechanical ventilation alarms and alarm fatigue} & 2019 & 29 & Yes \\
        \hline        
		\citet{west2014alarm} & \textit{Alarm Fatigue: A Concept Analysis} & 2014 & 28 & Yes \\
        \hline 
		\citet{horkan2014alarm} & \textit{Alarm fatigue and patient safety} & 2014 & 35 & No \\
		\hline
		\citet{blake2014effect} & \textit{The effect of alarm fatigue on the work environment} & 2014 & 14 & No \\
		\hline
		\citet{petersen2017assessment} & \textit{Assessment of clinical alarms influencing nurses' perceptions of alarm fatigue} & 2017 & 63 & No \\
		\hline
		\citet{baillargeon2013alarm} & \textit{Alarm fatigue: A risk assessment} & 2013 & 20 & Yes \\
		\hline
		\citet{welch2012alarm} & \textit{Alarm fatigue hazards: the sirens are calling} & 2012 & 46 & Yes \\
		\hline
		\citet{nyarko2023effect} & \textit{The effect of educational interventions in managing nurses' alarm fatigue: An integrative review} & 2023 & 8 & Yes \\
		\hline
		
	    \caption{Google scholar search results for `"alarm fatigue" evidence based practice' as of 14 March 2024} 
	\end{longtable}

    \chapter{Methodology}
    
    \section{Overview}
    
    \section{Research Questions}
    
    I concentrated on the following research questions:
    
    \begin{itemize}
    	\item What is `Alarm Fatigue'?
    	\item How to detect the onset of Alarm Fatigue?
    	\item What methods are used to overcome Alarm Fatigue?
    	\item What methods and systems are proven to prevent the onset of Alarm Fatigue?
    	\item Are there any specific aspects from the above questions that are applicable to IT?
    \end{itemize}

	\chapter{Monitor alarm fatigue: an integrative review}

\section{Overview}

\citet{cvach2012monitor} reviews the situation between 2000 and 2011. The then prevalent definition of \textit{Alarm Fatigue} as `...the lack of response due to excessive numbers of alarms resulting in sensory overload and desensitization...' (p.269) remains the same today. \\

The overall strategy to overcome \textit{Alarm Fatigue} is to devise and implement an alarm management policy, usually through an interdisciplinary alarm management committee \citep[p.273]{cvach2012monitor}.

\section{Abstract}

The abstract from \textit{Monitor alarm fatigue: an integrative review} \citep[p.268]{cvach2012monitor} is:

\begin{quotation}
	Alarm fatigue is a national problem and the number one medical device technology hazard in 2012. The problem of alarm desensitization is multifaceted and related to a high false alarm rate, poor positive predictive value, lack of alarm standardization, and the number of alarming medical devices in hospitals today. This integrative review synthesizes research and non-research findings published between 1/1/2000 and 10/1/2011 using The Johns Hopkins Nursing Evidence-Based Practice model. Seventy-two articles were included. Research evidence was organized into five main themes: excessive alarms and effects on staff; nurse's response to alarms; alarm sounds and audibility; technology to reduce false alarms; and alarm notification systems. Non-research evidence was divided into two main themes: strategies to reduce alarm desensitization, and alarm priority and notification systems. Evidence-based practice recommendations and gaps in research are summarized.
\end{quotation}

\section{What is `Alarm Fatigue'?}

A more general term is \textit{Alarm Hazard} which encompasses \textit{Alarm Fatigue}: \\

\begin{quote}
	“Alarm hazards” is the number one health technology hazard for 2012. Such hazards include inappropriate alarm modification, alarm fatigue, modifying alarms without restoring them to their original settings, and improperly relaying alarm signals to the appropriate person. (p.268)
\end{quote}

Later on, \textit{Alarm Fatigue} is defined as `...the lack of response due to excessive numbers of alarms resulting in sensory overload and desensitization...' (p.269).

\section{How to detect the onset of Alarm Fatigue?}

The detection of the onset of Alarm Fatigue is not considered.

\section{What methods are used to overcome Alarm Fatigue?}

The general response appears to organisational covering the following aspects: \\

\begin{itemize}
	\item Alarm risk assessment
	\item Alarm reduction
	\item Alarm management policy and committee
	\item Alarm accountability
	\item Alarm data
	\item Training
\end{itemize}

\citet[p.273]{cvach2012monitor} writes: \\

\begin{quotation}
	Organizations committed to finding solutions have formed interdisciplinary alarm management committees to conduct an alarm risk assessment and explore strategies for alarm reduction. An alarm management policy is essential to define alarm accountability. Alarm data informs proper settings for unit default parameter limits, assists in determining alarm prioritization, and reduces alarm fatigue. Each unit must be analyzed to determine the proper alarm management strategy. It is difficult to apply a “one-size-fits-all” approach to alarm management for all types of monitored units. Initial and ongoing training on alarming devices is recommended.
\end{quotation}

\section{What methods and systems are proven to prevent the onset of Alarm Fatigue?}

\citet[p.277]{cvach2012monitor} gives the state of evidence-based practice recommendations as of 2012:\\

\begin{quotation}
	To decrease monitor alarm fatigue, the following strategies are recommended:
	\begin{itemize}
		\item \textbf{Technology}
		\begin{itemize}
			\item Smart alarms, which take into account multiple parameters, rate of change and signal quality, can reduce the number of false alarms.
			\item Alarm technology that incorporates short delays can decrease the number of ignored or ineffective alarms caused by patient manipulation.
			\item Standardizing alarm sounds may be an effective way to reduce the number of alarms that staff must learn.
			\item Animated steps on the monitoring equipment for troubleshooting alarms would be helpful in assuring best practice with equipment.
		\end{itemize}
		\item \textbf{Hospital}
		\begin{itemize}
			\item Hospitals should engage an interdisciplinary alarm management committee to conduct an alarm risk assessment and explore strategies for alarm reduction.
			\item Hospitals should develop alarm setting and response protocols.
			\item Activated alarms should be set to actionable limits and levels.
			\item Staffing model should consider that alarm response time is a function of primary task workload; as workload increases, time to alarm response increases, and alarm task performance gets worse.
			\item Alarm enhancement technology provides additional means to deliver alarm signals from monitors to caregivers. These technologies may include pagers, phones, and auxiliary displays such as waveform screens. Use of alarm notification systems that provide context to the care provider and closed-loop communication is recommended.
			\item Investment in initial and ongoing training on alarming devices. Clinical competency that reflects institutional policy assures care provider skill with physiologic monitoring. Training should mimic the clinical environment where the device is used.
			\item To reduce patient and staff stress symptoms, noise reduction strategies should be employed.
		\end{itemize}
		\item \textbf{Caregiver}
		\begin{itemize}
			\item Staff could avoid false alarms by suspending alarms for a short time period prior to patient manipulation.
			\item Adjustment of alarms to patients' actual needs ensures that alarms are valid and provides an early warning to potential critical situations.
			\item Proper skin preparation and replacing ECG leads and electrodes routinely decreases false alarms.
			\item Documentation of alarm parameters in the medical record is an effective intervention for improving alarm adjustment compliance.
		\end{itemize}
	\end{itemize}
\end{quotation}
	\chapter{Diagnosing and Treating “Alarm Fatigue”: Pragmatic and Evidence-Based Approaches Needed}

\section{Overview}

\citet{rayo2016diagnosing} says that \textit{Alarm Fatigue} is an ill-defined term (p.291). Instead, a triplet of metrics is used to gauge alarm performance. Changes to alarm thresholds by nurses to reduce the incidence of non-actionable alarms is the main thrust to improve alarm performance. However, \textit{alarm performance} is not defined.

\section{Abstract}

No abstract is provided as this is a two (2) page report.

\section{What is `Alarm Fatigue'?}

\citet[p.291]{rayo2016diagnosing} says that \textit{Alarm Fatigue} `... is ill-defined, encompassing myriad observed symptoms and theories of underlying etiology.' \\

\section{How to detect the onset of Alarm Fatigue?}

There is no direct method for detection of the onset of \textit{Alarm Fatigue}, but \citet[p.292]{rayo2016diagnosing} writes: \\

\begin{quotation}
	As McGrath et al. note in their article, the Dartmouth-Hitchcock team’s choice of multiple inexpensive but indirect predictors of alarm system performance was a reflection of the need
	to trade off pragmatism and precision. Directly measuring informativeness, detectability, workload, and other high-impact
	characteristics can be expensive, particularly when trying to
	measure the prevalence of actionable and nonactionable alarms.
	Instead, they measured alarm duration, alarm rate, and clinician perception of the alarms. Alarm duration is a moderate
	to strong proxy for alarm response time, which is highly associated with positive predictive value, trust, and perceived importance. Individually, alarm rate and clinician perception are
	each weak predictors of alarm system performance, with studies showing poor associations between alarm rate and informativeness and alarm rate and clinician perception. Using all of
	these proxies together, however, affords more confidence than
	any of their individual results. Triangulating the results of weak
	predictors has been found to be a strong strategy in trading off
	pragmatism and evidential strength.
\end{quotation}

Three (3) metrics were collected: two (2) quantitative, and one (1) qualitative. All three (3) metrics need to be considered together in order to gauge alarm performance (which is not defined): \\

\begin{enumerate}
	\item Alarm duration
	\item Alarm rate
	\item Clinician perception
\end{enumerate}

No thresholds are given for these metrics. \\

Earlier, \citet[p.291]{rayo2016diagnosing} writes: \\

\begin{quotation}
	McGrath et al. have taken great pains to identify and improve
	multiple evidence-based measures of their alarm system, which
	would now include this new surveillance monitoring program.
	They optimized each within the realistic constraints of their organization. They addressed many of the alarm aspects shown
	to have high impact on \textbf{alarm performance} in the human factors engineering and applied psychology literature, as shown in
	Table 1 (above). They similarly made decisions about the visual
	and auditory characteristics of each new alarm to improve detection. They introduced new alarm delays, escalation paths, default alarm settings, patient education, new sensor placements,
	and more flexible alarm personalization policies to improve informativeness, which has repeatedly been found to be the best predictor of alarm response and therefore \textbf{alarm system performance}. Perhaps most notably, they created clear guidelines
	that provided nurses more authority (but not unlimited authority) to change patient alarm thresholds so that the nurses could
	more easily tailor the alarm settings to their patients’ dynamic
	condition, thereby reducing nonactionable alarms. Finally, they
	conducted detailed work-flow task analyses to reduce overall
	workload. For each of these truly high-impact alarm characteristics, validated models were used to prioritize interventions and
	predict their impacts. \\
	\begin{flushright}
		\textbf{Emphasis Mine}
	\end{flushright}
\end{quotation}

The primary predictor of \textbf{alarm system performance} is \textit{alarm response time}. It appears that the primary components of \textbf{alarm system performance} are: \\

\begin{enumerate}
	\item Alarm response time
	\item Actionable versus non-actionable alarms
\end{enumerate}

I would guess that an alarm system performs well if all actionable alarms are attended to within a reasonable time to prevent things getting worse. \\

\section{What Methods are used to overcome Alarm Fatigue?}

\citet[p.291]{rayo2016diagnosing} write: \\

\begin{quotation}
	...they created clear guidelines
	that provided nurses more authority (but not unlimited authority) to change patient alarm thresholds so that the nurses could
	more easily tailor the alarm settings to their patients’ dynamic
	condition, thereby reducing nonactionable alarms. Finally, they
	conducted detailed work-flow task analyses to reduce overall
	workload. For each of these truly high-impact alarm characteristics, validated models were used to prioritize interventions and
	predict their impacts.
\end{quotation}

\section{What Methods and Systems are proven to prevent the onset of Alarm Fatigue?}

Since the onset of \textit{Alarm Fatigue} is not recognised explicitly, there is no direct intervention, but, rather, the maintenance of a safe alarm environment is the key. \\

In other words, it would appear there is a qualitative monitoring using the following factors, and intervention is deemed appropriate if the combined effect is too high: \\

\begin{enumerate}
	\item Alarm duration
	\item Alarm rate
	\item Clinician perception
\end{enumerate}


	\chapter{Determining the impact of an alarm management program on alarm fatigue among ICU and Telemetry RNs: An evidence based research project}

\section{Overview}

\citet{dee2022determining} is a literature review of six (6) articles. Their conclusion is that there is no proven method of reducing \textit{Alarm Fatigue}, but some sort of \textit{Alarm Management} appears to be promising.

\section{Keywords}

The keywords for \citet{dee2022determining} are:

\begin{itemize}
	\item alarm fatigue
	\item registered nurse
	\item telemetry
	\item icu nurses
	\item alarms
\end{itemize}

\section{Abstract}

The abstract from \citet{dee2022determining} is:

\begin{quotation}
	This evidence-based research project provides an appraisal of current research on how an alarm management program impacts
	alarm fatigue among registered nurses (RNs) in both intensive care units (ICUs) and telemetry units. Alarm fatigue is a major problem recognized by both the American Association of Critical-Care Nurses (AACN) and the Joint Commission. RNs are the primary caretakers of critically ill patients in ICUs and telemetry units and therefore are at the greatest risk for alarm fatigue. \\
	The researchers performed an evidence synthesis to determine the impact of an alarm management program on alarm fatigue
	among ICU and telemetry RNs. A literature search was conducted using scientific databases such as PubMed, CINAHL, Trip,
	Cochrane Review, and Google Scholar. Our search strategy included the following terms: adult registered nurse, inpatient registered nurse, ICU registered nurses, RNs, Nurse Practitioners, alarm fatigue, alarm management strategy, education, cardiac monitor
	alarm, alarm strategies, alarm bundle, telemetry alarm, and cardiac monitor. Any studies involving the pediatric population, pulse
	oximeter alarms, and ventilator alarms were excluded. Due to the lack of available randomized control trials and cohort studies,
	the authors included two quality improvement (QI) projects. Finally, six studies were taken into consideration for review. The
	authors appraised each of the six articles using the Critical Appraisal Skills Programme Checklist (CASP) Tool. This tool allowed
	the authors to synthesize information based on the outcomes and determine the level of the evidence of each article in order to
	make evidence-based practice recommendations on implementing alarm management programs. \textbf{Conclusion: Despite extensive
	literature highlighting the astronomical prevalence of alarm fatigue in RNs, there was a lack of current high-quality data related to
	implementing alarm management programs.} Therefore, more research is needed to support the utilization of alarm management
	programs in ICUs and telemetry units to improve alarm fatigue among RNs. \\
	\begin{flushright}
		\textbf{(Emphasis Mine)}
	\end{flushright}
\end{quotation}

\section{What is `Alarm Fatigue'?}

`Alarm fatigue is defined as desensitization and apathy of
healthcare providers to the sound of an overwhelming
number of repetitive or simultaneous alarms (Lewis \&
Oster, 2019).' \citep[p.2]{dee2022determining}. \\

\section{How to Detect the Onset of Alarm Fatigue?}

No mention of the onset of \textit{Alarm Fatigue} is made in the article.

\section{What Methods are used to Overcome Alarm Fatigue?}

Generally, the method is to implement an \textit{Alarm Management} system which is ill-defined, but includes some sort of structured training. \\

`The primary intervention to reduce alarm
fatigue is through alarm management (Lewis \& Oster,
2019).' \citep[p.2]{dee2022determining} \\

\citet[p.2]{dee2022determining} writes:

\begin{quotation}
	\textbf{Alarm management can be accomplished through a combination of evidence-based interventions, including educational programs, development of policies and procedures,
		reducing over-monitoring of patients, and customizing
		alarm parameters.} Literature suggests that implementing alarm
	management programs decreases the likelihood of alarm
	fatigue in nurses in the inpatient setting. These programs may
	include daily reassessment of need for each monitoring alarm,
	proper skin preparation prior to applying adhesive-based monitoring technology, and frequency of changing or recalibrating
	monitoring devices (Lewis \& Oster, 2019) This has the potential
	to improve patient safety and significantly reduce sentinel
	events in the inpatient population.
	\begin{flushright}
		\textbf{(Emphasis Mine)}
	\end{flushright}
\end{quotation}

\section{What Methods and Systems are Proven to Prevent the Onset of Alarm Fatigue?}

\citet[p.13]{dee2022determining} concludes that there are no proven methods to prevent \textit{Alarm Fatigue}, but deem \textit{Alarm Management} to the best hope. Their conclusion is: \\

\begin{quotation}
	Despite extensive literature highlighting the astronomical
	prevalence of alarm fatigue in RNs, there was a lack of
	data related to implementing alarm management programs.
	Therefore, more research is needed to support the utilization
	of alarm management programs in ICUs and telemetry units
	to improve alarm fatigue among RNs.
\end{quotation}

\section{Other Comments}

This was a meta-analysis of six (6) articles that included:

\begin{enumerate}
	\item \citet{bi2020effects}
\end{enumerate}

Only one (1) overlaps between their review and mine.

\subsection{Author's Comments on Bi 2020}

\citet[pp.5-6]{dee2022determining} write of \citet{bi2020effects}: \\

\begin{quotation}
	Strengths- using the theory of planned behavior to help decrease alarm fatigue and lowering the number of alarms. \\
	Limitations include generalizability as this was only conducted in the ICU. \\
	Inability to blind the control group due to practical reasons is a limitation due to possible contamination of control group. \\
	Total alarms was recorded but nonactionable/crisis alarms were judged by experts and are subject to human error and omissions. \\
	Time- this was a short study with no longer term followup.
\end{quotation}

	\chapter{Evidence-based Initiative to Reduce Alarm Fatigue}

\section{Overview}

\citet{gorisek2021evidence} restricts the definition of \textit{Alarm Fatigue} to critical alarms. Their solution is to implement a Quality Improvement project.

\section{Keywords}

No keywords were provided.

\section{Abstract}

Only the abstract from \textit{An evidence-based initiative to reduce alarm fatigue in a burn intensive care unit} is available for free online: \\

\begin{quotation}
	
	\textbf{Background} \\
	
	Alarm fatigue occurs when nurses are exposed to multiple alarms of mixed significance and become desensitized to alarms to the point that a critical alarm may receive no response or a delayed response. In burn intensive care units, reducing the risk of alarm fatigue is uniquely challenging because of the critically ill patient population and the nature of burn skin injuries. Nurses and the interdisciplinary team can become fatigued and desensitized to alarms, decreasing response rates for necessary interventions. \\
	
	\textbf{Objective} \\
	
	To decrease the risk of alarm fatigue by using an initiative designed to reduce nonactionable and false alarms in a burn intensive care unit. \\
	
	\textbf{Methods} \\
	
	Baseline data (alarm count per patient-day by alarm type) were collected for 1 month before education and implementation of evidence-based interventions. Data were collected every 6 months for 2 years. \\
	
	\textbf{Interventions} \\
	
	A series of interventions included raising awareness of the risks associated with alarm fatigue, customizing alarm parameters and default settings, providing education on electrode placement and daily electrode changes, using physical reminders, and consistently sharing alarm data. The education, delivered in modules, aligned with the evidence-based interventions. \\
	
	\textbf{Results} \\
	
	Preintervention baseline data were compared to postintervention data at 6, 12, 18, and 24 months. The results showed a significantly sustained reduction (P \textless .001) in total alarm rate over time. \\
	
	\textbf{Conclusion} \\
	
	A quality improvement initiative based on evidence-based practice can contribute to a sustainable reduction in nonactionable and false alarms, ultimately improving patient safety.
	
\end{quotation}

\section{What is `Alarm Fatigue'?}

`Alarm fatigue occurs when nurses are exposed to multiple alarms of mixed significance and become desensitized to alarms to the point that a critical alarm may receive no response or a delayed response.'

\section{How to Detect the Onset of Alarm Fatigue?}

No methods of detecting the onset of \textit{Alarm Fatigue} were mentioned in the abstract.

\section{What Methods are used to Overcome Alarm Fatigue?}

\begin{quotation}
	A series of interventions included raising awareness of the risks associated with alarm fatigue, customizing alarm parameters and default settings, providing education on electrode placement and daily electrode changes, using physical reminders, and consistently sharing alarm data. The education, delivered in modules, aligned with the evidence-based interventions.
\end{quotation}

\section{What Methods and Systems are Proven to Prevent the Onset of Alarm Fatigue?}

\begin{quotation}
	A quality improvement initiative based on evidence-based practice can contribute to a sustainable reduction in nonactionable and false alarms, ultimately improving patient safety.
\end{quotation}



	\chapter{An evidence-based approach to reduce nuisance alarms and alarm fatigue}

\section{Overview}

\citet{welch2011evidence} claims that \textit{Alarm Fatigue} can be eliminated by tuning the alarm thresholds and delay settings. Since the author is a manufacturer, the article is biased towards what advice these settings should be based on the data they collected. Their emphasis is on reducing the incidence of \textit{false alarms} and \textit{non-actionable alarms}.

\section{Keywords}

No keywords are provided.

\section{Abstract}

The abstract from \citet{welch2012alarm} is: \\

\begin{quotation}
	To help clinicians make evidence-based
	decisions about where to program alarm
	settings, Masimo Corp. based in Irvine, CA
	conducted an analysis of 32 million pulse
	oximetry (SpO2) data points from 10 hospital
	general post-surgical care areas. Each hospital
	was equipped with a Masimo Patient
	SafetyNetTM remote monitoring and clinician
	notification system, which continuously
	captures and stores time-stamped SpO2 data
	with a one-second resolution. This paper
	reports on the results of a retrospective analysis
	conducted by the company to determine the
	incidence of alarms at various alarm threshold
	and delay settings.
\end{quotation}

\section{What is `Alarm Fatigue'?}

\cite[p.49]{welch2011evidence} gives the following definitions: \\

\begin{quotation}
	\textbf{Key Terms} \\
	\textbf{Actionable Alarms}: Alarms that
	require a response to bedside and
	therapeutic intervention to avoid an
	adverse event. \\
	\textbf{Alarm Fatigue}: Failure to recognize
	and respond to true alarms that
	require bedside clinical intervention
	as a result of high occurrence
	of alarms. \\
	\textbf{False Alarms}: Alarms due to artifact that produce false data. \\
	\textbf{Nonactionable Alarms}: True
	alarms that do not require patient
	therapeutic intervention. \\
	\textbf{Nuisance Alarms}: The high
	occurrence of clinically non-actionable alarms. \\
	\textbf{True Alarms}: Alarms
	that represent true
	and accurate
	physiologic data
\end{quotation}

\section{How to Detect the Onset of Alarm Fatigue?}

No mention of \textit{Alarm Fatigue} onset is made.

\section{What Methods are used to Overcome Alarm Fatigue?}

The nebulous term, \textit{Alarm Management}, is used. \citet[p.51]{welch2011evidence} writes:

\begin{quotation}
	Eliminating alarm fatigue is a shared
	responsibility between clinicians, clinical/biomedical engineers and industry. Clinicians
	determine policies regarding which patients are
	monitored and what alarms are set. Biomedical
	professionals support clinicians by selecting,
	implementing, and maintaining the best and
	most cost-effective technology solutions.
	Both depend on industry to provide
	technology solutions.
\end{quotation}

\section{What Methods and Systems are Proven to Prevent the Onset of Alarm Fatigue?}

Since the author believes \textit{Alarm Fatigue} can be eliminated, there is no discussion of how to prevent the onset of the syndrome.


	\chapter{Alarm Fatigue: A Concept Analysis}

\section{Overview}

\textit{Alarm Fatigue: A Concept Analysis} (\cite{west2014alarm}) provides a very good definition of alarm fatigue in a clinical setting.

\section{Keywords}

\cite{west2014alarm} uses these keywords: concept analysis, alarm fatigue, nursing, technology, distractions

\section{Definition}

\cite{west2014alarm} quotes \cite{mccartney2012clinical} in the definition:

\begin{quote}
	Alarm fatigue, defined in the literature is the desensitization of a clinician to an alarm stimulus that results from sensory overload causing the response of an alarm to be delayed or missed \cite{mccartney2012clinical}.
\end{quote}

This definition is expanded as follows:

\begin{quotation}
	Therefore, alarm fatigue encompasses three defining attributes: 
	\begin{itemize}
		\item an environment with excessive and repeated situations;
		\item a lessened motivation and interest in surroundings;
		\item and a diminished capacity for physical and mental work.
	\end{itemize}
\end{quotation}

The cycle of alarm fatigue is described as:

\begin{quotation}
	Thus, the antecedents of alarm fatigue are:
	\begin{itemize}
		\item involvement of a healthcare professional;
		\item the ability to subjectively evaluate feelings;
		\item a patient care environment with excessive stimuli
	\end{itemize}
	The consequences of alarm fatigue are:
	\begin{itemize}
		\item a lessened capacity to give a normal response to a signal;
		\item a significant clinical event missed or ignored that could lead to a potentially harmful patient situation;
		\item limited perception of the clinical significance of the alarm signal. \\
	\end{itemize}
\end{quotation}

	\section{Research Outcomes of Implementing CEASE}

\textit{Research Outcomes of Implementing CEASE: An Innovative, Nurse-Driven, Evidence-Based, Patient-Customized Monitoring Bundle to Decrease Alarm Fatigue in the Intensive Care Unit/Step-down Unit}

\cite{lewis2019research}

	
	\chapter{Conclusion}
	
	\bibliography{main}

\end{document}