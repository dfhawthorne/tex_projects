\chapter{Clinical Alarm Management}

\section{Overview}

\citet{mccartney2012clinical}[p.202] defines \textbf{alarm fatigue} as occurring `...when a clinician becomes desensitized to the alarm stimulus and ignores the alarm, overrides the alarm, or has a delayed response to the alarm.' The cause is identified as `[c]onstant exposure to false positive alarms, alarms that do not require an intervention, and the perpetual interruption of alarms...' \textbf{Alarm fatigue} is distinguished from \textbf{selective attention} in which a clinician focuses on `...one alarm while ignoring others.' In case, an alarm is not being ignored: other alarms are  ignored. \\

\section{Author}

\subsection{Patricia Robin McCartney}

PR McCartney has no published Google Scholar profile. \\

Patricia Robin McCartney is the Director of Nursing Research and Georgetown Scholars Clinical Instructor, Washington Hospital Center, Waashington, DC, Professor Emertita, State University of New York at Buffalo, and an Editorial Board Member of MCN (The American Journal of Maternal/Child Nursing).

\subsection{Extract}

\citet{mccartney2012clinical}[p.202] reports on a 2012 summit about alarm management: \\

\begin{quotation}
	Constant exposure to false positive alarms, alarms that do not require an intervention, and the perpetual interruption of alarms lead to alarm fatigue. Alarm fatigue occurs when a clinician becomes desensitized to the alarm stimulus and ignores the alarm, overrides the alarm, or has a delayed response to the alarm. An overload of alarms (alarm burden) can lead to clinician selective attention and a focus one alarm while ignoring others. Alarm fatigue and selective attention have potential for patient harm. Some proposed remedies for alarm fatigue are customizing alarms to the individual patient's physiology and prioritizing alarms that present to the nurse.
\end{quotation}

\subsection{Definition}

`Alarm fatigue occurs when a clinician becomes desensitized to the alarm stimulus and ignores the alarm, overrides the alarm, or has a delayed response to the alarm.' \\

Selective attention is distinguished from alarm fatigue.

\subsection{Cause}

`Constant exposure to false positive alarms, alarms that do not require an intervention, and the perpetual interruption of alarms lead to alarm fatigue.'
