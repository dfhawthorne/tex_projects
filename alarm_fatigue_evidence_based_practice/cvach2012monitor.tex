\chapter{Monitor alarm fatigue: an integrative review}

\section{Overview}

\citet{cvach2012monitor} reviews the situation between 2000 and 2011. The then prevalent definition of \textit{Alarm Fatigue} as `...the lack of response due to excessive numbers of alarms resulting in sensory overload and desensitization...' (p.269) remains the same today. \\

The overall strategy to overcome \textit{Alarm Fatigue} is to devise and implement an alarm management policy, usually through an interdisciplinary alarm management committee \citet[p.273]{cvach2012monitor}.

\section{Abstract}

The abstract from \textit{Monitor alarm fatigue: an integrative review} \citet[p.268]{cvach2012monitor} is:

\begin{quotation}
	Alarm fatigue is a national problem and the number one medical device technology hazard in 2012. The problem of alarm desensitization is multifaceted and related to a high false alarm rate, poor positive predictive value, lack of alarm standardization, and the number of alarming medical devices in hospitals today. This integrative review synthesizes research and non-research findings published between 1/1/2000 and 10/1/2011 using The Johns Hopkins Nursing Evidence-Based Practice model. Seventy-two articles were included. Research evidence was organized into five main themes: excessive alarms and effects on staff; nurse's response to alarms; alarm sounds and audibility; technology to reduce false alarms; and alarm notification systems. Non-research evidence was divided into two main themes: strategies to reduce alarm desensitization, and alarm priority and notification systems. Evidence-based practice recommendations and gaps in research are summarized.
\end{quotation}

\section{What is `Alarm Fatigue'?}

A more general term is \textit{Alarm Hazard} which encompasses \textit{Alarm Fatigue}: \\

\begin{quote}
	“Alarm hazards” is the number one health technology hazard for 2012. Such hazards include inappropriate alarm modification, alarm fatigue, modifying alarms without restoring them to their original settings, and improperly relaying alarm signals to the appropriate person. (p.268)
\end{quote}

Later on, \textit{Alarm Fatigue} is defined as `...the lack of response due to excessive numbers of alarms resulting in sensory overload and desensitization...' (p.269).

\section{How to detect the onset of Alarm Fatigue?}

The detection of the onset of Alarm Fatigue is not considered.

\section{What methods are used to overcome Alarm Fatigue?}

The general response appears to organisational covering the following aspects: \\

\begin{itemize}
	\item Alarm risk assessment
	\item Alarm reduction
	\item Alarm management policy and committee
	\item Alarm accountability
	\item Alarm data
	\item Training
\end{itemize}

\citet[p.273]{cvach2012monitor} writes: \\

\begin{quotation}
	Organizations committed to finding solutions have formed interdisciplinary alarm management committees to conduct an alarm risk assessment and explore strategies for alarm reduction. An alarm management policy is essential to define alarm accountability. Alarm data informs proper settings for unit default parameter limits, assists in determining alarm prioritization, and reduces alarm fatigue. Each unit must be analyzed to determine the proper alarm management strategy. It is difficult to apply a “one-size-fits-all” approach to alarm management for all types of monitored units. Initial and ongoing training on alarming devices is recommended.
\end{quotation}

\section{What methods and systems are proven to prevent the onset of Alarm Fatigue?}

\citep[p.277]{cvach2012monitor} gives the state of evidence-based practice recommendations as of 2012:\\

\begin{quotation}
	To decrease monitor alarm fatigue, the following strategies are recommended:
	\begin{itemize}
		\item \textbf{Technology}
		\begin{itemize}
			\item Smart alarms, which take into account multiple parameters, rate of change and signal quality, can reduce the number of false alarms.
			\item Alarm technology that incorporates short delays can decrease the number of ignored or ineffective alarms caused by patient manipulation.
			\item Standardizing alarm sounds may be an effective way to reduce the number of alarms that staff must learn.
			\item Animated steps on the monitoring equipment for troubleshooting alarms would be helpful in assuring best practice with equipment.
		\end{itemize}
		\item \textbf{Hospital}
		\begin{itemize}
			\item Hospitals should engage an interdisciplinary alarm management committee to conduct an alarm risk assessment and explore strategies for alarm reduction.
			\item Hospitals should develop alarm setting and response protocols.
			\item Activated alarms should be set to actionable limits and levels.
			\item Staffing model should consider that alarm response time is a function of primary task workload; as workload increases, time to alarm response increases, and alarm task performance gets worse.
			\item Alarm enhancement technology provides additional means to deliver alarm signals from monitors to caregivers. These technologies may include pagers, phones, and auxiliary displays such as waveform screens. Use of alarm notification systems that provide context to the care provider and closed-loop communication is recommended.
			\item Investment in initial and ongoing training on alarming devices. Clinical competency that reflects institutional policy assures care provider skill with physiologic monitoring. Training should mimic the clinical environment where the device is used.
			\item To reduce patient and staff stress symptoms, noise reduction strategies should be employed.
		\end{itemize}
		\item \textbf{Caregiver}
		\begin{itemize}
			\item Staff could avoid false alarms by suspending alarms for a short time period prior to patient manipulation.
			\item Adjustment of alarms to patients' actual needs ensures that alarms are valid and provides an early warning to potential critical situations.
			\item Proper skin preparation and replacing ECG leads and electrodes routinely decreases false alarms.
			\item Documentation of alarm parameters in the medical record is an effective intervention for improving alarm adjustment compliance.
		\end{itemize}
	\end{itemize}
\end{quotation}