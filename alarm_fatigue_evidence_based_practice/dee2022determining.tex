\chapter{Determining the impact of an alarm management program on alarm fatigue among ICU and Telemetry RNs: An evidence based research project}

\section{Overview}

\citet{dee2022determining} is a literature review of six (6) articles. Their conclusion is that there is no proven method of reducing \textit{Alarm Fatigue}, but some sort of \textit{Alarm Management} appears to be promising.

\section{Keywords}

The keywords for \citet{dee2022determining} are:

\begin{itemize}
	\item alarm fatigue
	\item registered nurse
	\item telemetry
	\item icu nurses
	\item alarms
\end{itemize}

\section{Abstract}

The abstract from \citet{dee2022determining} is:

\begin{quotation}
	This evidence-based research project provides an appraisal of current research on how an alarm management program impacts
	alarm fatigue among registered nurses (RNs) in both intensive care units (ICUs) and telemetry units. Alarm fatigue is a major problem recognized by both the American Association of Critical-Care Nurses (AACN) and the Joint Commission. RNs are the primary caretakers of critically ill patients in ICUs and telemetry units and therefore are at the greatest risk for alarm fatigue. \\
	The researchers performed an evidence synthesis to determine the impact of an alarm management program on alarm fatigue
	among ICU and telemetry RNs. A literature search was conducted using scientific databases such as PubMed, CINAHL, Trip,
	Cochrane Review, and Google Scholar. Our search strategy included the following terms: adult registered nurse, inpatient registered nurse, ICU registered nurses, RNs, Nurse Practitioners, alarm fatigue, alarm management strategy, education, cardiac monitor
	alarm, alarm strategies, alarm bundle, telemetry alarm, and cardiac monitor. Any studies involving the pediatric population, pulse
	oximeter alarms, and ventilator alarms were excluded. Due to the lack of available randomized control trials and cohort studies,
	the authors included two quality improvement (QI) projects. Finally, six studies were taken into consideration for review. The
	authors appraised each of the six articles using the Critical Appraisal Skills Programme Checklist (CASP) Tool. This tool allowed
	the authors to synthesize information based on the outcomes and determine the level of the evidence of each article in order to
	make evidence-based practice recommendations on implementing alarm management programs. \textbf{Conclusion: Despite extensive
	literature highlighting the astronomical prevalence of alarm fatigue in RNs, there was a lack of current high-quality data related to
	implementing alarm management programs.} Therefore, more research is needed to support the utilization of alarm management
	programs in ICUs and telemetry units to improve alarm fatigue among RNs. \\
	\begin{flushright}
		\textbf{(Emphasis Mine)}
	\end{flushright}
\end{quotation}

\section{What is `Alarm Fatigue'?}

`Alarm fatigue is defined as desensitization and apathy of
healthcare providers to the sound of an overwhelming
number of repetitive or simultaneous alarms (Lewis \&
Oster, 2019).' \citep[p.2]{dee2022determining}. \\

\section{How to Detect the Onset of Alarm Fatigue?}

No mention of the onset of \textit{Alarm Fatigue} is made in the article.

\section{What Methods are used to Overcome Alarm Fatigue?}

Generally, the method is to implement an \textit{Alarm Management} system which is ill-defined, but includes some sort of structured training. \\

`The primary intervention to reduce alarm
fatigue is through alarm management (Lewis \& Oster,
2019).' \citep[p.2]{dee2022determining} \\

\citet[p.2]{dee2022determining} writes:

\begin{quotation}
	\textbf{Alarm management can be accomplished through a combination of evidence-based interventions, including educational programs, development of policies and procedures,
		reducing over-monitoring of patients, and customizing
		alarm parameters.} Literature suggests that implementing alarm
	management programs decreases the likelihood of alarm
	fatigue in nurses in the inpatient setting. These programs may
	include daily reassessment of need for each monitoring alarm,
	proper skin preparation prior to applying adhesive-based monitoring technology, and frequency of changing or recalibrating
	monitoring devices (Lewis \& Oster, 2019) This has the potential
	to improve patient safety and significantly reduce sentinel
	events in the inpatient population.
	\begin{flushright}
		\textbf{(Emphasis Mine)}
	\end{flushright}
\end{quotation}

\section{What Methods and Systems are Proven to Prevent the Onset of Alarm Fatigue?}

\citet[p.13]{dee2022determining} concludes that there are no proven methods to prevent \textit{Alarm Fatigue}, but deem \textit{Alarm Management} to the best hope. Their conclusion is: \\

\begin{quotation}
	Despite extensive literature highlighting the astronomical
	prevalence of alarm fatigue in RNs, there was a lack of
	data related to implementing alarm management programs.
	Therefore, more research is needed to support the utilization
	of alarm management programs in ICUs and telemetry units
	to improve alarm fatigue among RNs.
\end{quotation}

\section{Other Comments}

This was a meta-analysis of six (6) articles that included:

\begin{enumerate}
	\item \citet{bi2020effects}
\end{enumerate}

Only one (1) overlaps between their review and mine.

\subsection{Author's Comments on Bi 2020}

\citet[pp.5-6]{dee2022determining} write of \citet{bi2020effects}: \\

\begin{quotation}
	Strengths- using the theory of planned behavior to help decrease alarm fatigue and lowering the number of alarms. \\
	Limitations include generalizability as this was only conducted in the ICU. \\
	Inability to blind the control group due to practical reasons is a limitation due to possible contamination of control group. \\
	Total alarms was recorded but nonactionable/crisis alarms were judged by experts and are subject to human error and omissions. \\
	Time- this was a short study with no longer term followup.
\end{quotation}
