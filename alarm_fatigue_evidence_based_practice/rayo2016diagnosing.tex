\chapter{Diagnosing and Treating “Alarm Fatigue”: Pragmatic and Evidence-Based Approaches Needed}

\section{Overview}

\citet{rayo2016diagnosing} says that \textit{Alarm Fatigue} is an ill-defined term (p.291). Instead, a triplet of metrics is used to gauge alarm performance. Changes to alarm thresholds by nurses to reduce the incidence of non-actionable alarms is the main thrust to improve alarm performance. However, \textit{alarm performance} is not defined.

\section{Abstract}

No abstract is provided as this is a two (2) page report.

\section{What is `Alarm Fatigue'?}

\citet[p.291]{rayo2016diagnosing} says that \textit{Alarm Fatigue} `... is ill-defined, encompassing myriad observed symptoms and theories of underlying etiology.' \\

\section{How to detect the onset of Alarm Fatigue?}

There is no direct method for detection of the onset of \textit{Alarm Fatigue}, but \citet[p.292]{rayo2016diagnosing} writes: \\

\begin{quotation}
	As McGrath et al. note in their article, the Dartmouth-Hitchcock team’s choice of multiple inexpensive but indirect predictors of alarm system performance was a reflection of the need
	to trade off pragmatism and precision. Directly measuring informativeness, detectability, workload, and other high-impact
	characteristics can be expensive, particularly when trying to
	measure the prevalence of actionable and nonactionable alarms.
	Instead, they measured alarm duration, alarm rate, and clinician perception of the alarms. Alarm duration is a moderate
	to strong proxy for alarm response time, which is highly associated with positive predictive value, trust, and perceived importance. Individually, alarm rate and clinician perception are
	each weak predictors of alarm system performance, with studies showing poor associations between alarm rate and informativeness and alarm rate and clinician perception. Using all of
	these proxies together, however, affords more confidence than
	any of their individual results. Triangulating the results of weak
	predictors has been found to be a strong strategy in trading off
	pragmatism and evidential strength.
\end{quotation}

Three (3) metrics were collected: two (2) quantitative, and one (1) qualitative. All three (3) metrics need to be considered together in order to gauge alarm performance (which is not defined): \\

\begin{enumerate}
	\item Alarm duration
	\item Alarm rate
	\item Clinician perception
\end{enumerate}

No thresholds are given for these metrics. \\

Earlier, \citet[p.291]{rayo2016diagnosing} writes: \\

\begin{quotation}
	McGrath et al. have taken great pains to identify and improve
	multiple evidence-based measures of their alarm system, which
	would now include this new surveillance monitoring program.
	They optimized each within the realistic constraints of their organization. They addressed many of the alarm aspects shown
	to have high impact on \textbf{alarm performance} in the human factors engineering and applied psychology literature, as shown in
	Table 1 (above). They similarly made decisions about the visual
	and auditory characteristics of each new alarm to improve detection. They introduced new alarm delays, escalation paths, default alarm settings, patient education, new sensor placements,
	and more flexible alarm personalization policies to improve informativeness, which has repeatedly been found to be the best predictor of alarm response and therefore \textbf{alarm system performance}. Perhaps most notably, they created clear guidelines
	that provided nurses more authority (but not unlimited authority) to change patient alarm thresholds so that the nurses could
	more easily tailor the alarm settings to their patients’ dynamic
	condition, thereby reducing nonactionable alarms. Finally, they
	conducted detailed work-flow task analyses to reduce overall
	workload. For each of these truly high-impact alarm characteristics, validated models were used to prioritize interventions and
	predict their impacts. \\
	\begin{flushright}
		\textbf{Emphasis Mine}
	\end{flushright}
\end{quotation}

The primary predictor of \textbf{alarm system performance} is \textit{alarm response time}. It appears that the primary components of \textbf{alarm system performance} are: \\

\begin{enumerate}
	\item Alarm response time
	\item Actionable versus non-actionable alarms
\end{enumerate}

I would guess that an alarm system performs well if all actionable alarms are attended to within a reasonable time to prevent things getting worse. \\

\section{What Methods are used to overcome Alarm Fatigue?}

\citet[p.291]{rayo2016diagnosing} write: \\

\begin{quotation}
	...they created clear guidelines
	that provided nurses more authority (but not unlimited authority) to change patient alarm thresholds so that the nurses could
	more easily tailor the alarm settings to their patients’ dynamic
	condition, thereby reducing nonactionable alarms. Finally, they
	conducted detailed work-flow task analyses to reduce overall
	workload. For each of these truly high-impact alarm characteristics, validated models were used to prioritize interventions and
	predict their impacts.
\end{quotation}

\section{What Methods and Systems are proven to prevent the onset of Alarm Fatigue?}

Since the onset of \textit{Alarm Fatigue} is not recognised explicitly, there is no direct intervention, but, rather, the maintenance of a safe alarm environment is the key. \\

In other words, it would appear there is a qualitative monitoring using the following factors, and intervention is deemed appropriate if the combined effect is too high: \\

\begin{enumerate}
	\item Alarm duration
	\item Alarm rate
	\item Clinician perception
\end{enumerate}

