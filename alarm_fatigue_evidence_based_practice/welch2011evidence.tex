\chapter{An evidence-based approach to reduce nuisance alarms and alarm fatigue}

\section{Overview}

\citet{welch2011evidence} claims that \textit{Alarm Fatigue} can be eliminated by tuning the alarm thresholds and delay settings. Since the author is a manufacturer, the article is biased towards what advice these settings should be based on the data they collected. Their emphasis is on reducing the incidence of \textit{false alarms} and \textit{non-actionable alarms}.

\section{Keywords}

No keywords are provided.

\section{Abstract}

The abstract from \citet{welch2012alarm} is: \\

\begin{quotation}
	To help clinicians make evidence-based
	decisions about where to program alarm
	settings, Masimo Corp. based in Irvine, CA
	conducted an analysis of 32 million pulse
	oximetry (SpO2) data points from 10 hospital
	general post-surgical care areas. Each hospital
	was equipped with a Masimo Patient
	SafetyNetTM remote monitoring and clinician
	notification system, which continuously
	captures and stores time-stamped SpO2 data
	with a one-second resolution. This paper
	reports on the results of a retrospective analysis
	conducted by the company to determine the
	incidence of alarms at various alarm threshold
	and delay settings.
\end{quotation}

\section{What is `Alarm Fatigue'?}

\cite[p.49]{welch2011evidence} gives the following definitions: \\

\begin{quotation}
	\textbf{Key Terms} \\
	\textbf{Actionable Alarms}: Alarms that
	require a response to bedside and
	therapeutic intervention to avoid an
	adverse event. \\
	\textbf{Alarm Fatigue}: Failure to recognize
	and respond to true alarms that
	require bedside clinical intervention
	as a result of high occurrence
	of alarms. \\
	\textbf{False Alarms}: Alarms due to artifact that produce false data. \\
	\textbf{Nonactionable Alarms}: True
	alarms that do not require patient
	therapeutic intervention. \\
	\textbf{Nuisance Alarms}: The high
	occurrence of clinically non-actionable alarms. \\
	\textbf{True Alarms}: Alarms
	that represent true
	and accurate
	physiologic data
\end{quotation}

\section{How to Detect the Onset of Alarm Fatigue?}

No mention of \textit{Alarm Fatigue} onset is made.

\section{What Methods are used to Overcome Alarm Fatigue?}

The nebulous term, \textit{Alarm Management}, is used. \citet[p.51]{welch2011evidence} writes:

\begin{quotation}
	Eliminating alarm fatigue is a shared
	responsibility between clinicians, clinical/biomedical engineers and industry. Clinicians
	determine policies regarding which patients are
	monitored and what alarms are set. Biomedical
	professionals support clinicians by selecting,
	implementing, and maintaining the best and
	most cost-effective technology solutions.
	Both depend on industry to provide
	technology solutions.
\end{quotation}

\section{What Methods and Systems are Proven to Prevent the Onset of Alarm Fatigue?}

Since the author believes \textit{Alarm Fatigue} can be eliminated, there is no discussion of how to prevent the onset of the syndrome.

