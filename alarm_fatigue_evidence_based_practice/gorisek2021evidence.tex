\chapter{Evidence-based Initiative to Reduce Alarm Fatigue}

\textit{An evidence-based initiative to reduce alarm fatigue in a burn intensive care unit}

\begin{quotation}
	
	\textbf{Background} \\
	
	Alarm fatigue occurs when nurses are exposed to multiple alarms of mixed significance and become desensitized to alarms to the point that a critical alarm may receive no response or a delayed response. In burn intensive care units, reducing the risk of alarm fatigue is uniquely challenging because of the critically ill patient population and the nature of burn skin injuries. Nurses and the interdisciplinary team can become fatigued and desensitized to alarms, decreasing response rates for necessary interventions. \\
	
	\textbf{Objective} \\
	
	To decrease the risk of alarm fatigue by using an initiative designed to reduce nonactionable and false alarms in a burn intensive care unit. \\
	
	\textbf{Methods} \\
	
	Baseline data (alarm count per patient-day by alarm type) were collected for 1 month before education and implementation of evidence-based interventions. Data were collected every 6 months for 2 years. \\
	
	\textbf{Interventions} \\
	
	A series of interventions included raising awareness of the risks associated with alarm fatigue, customizing alarm parameters and default settings, providing education on electrode placement and daily electrode changes, using physical reminders, and consistently sharing alarm data. The education, delivered in modules, aligned with the evidence-based interventions. \\
	
	\textbf{Results} \\
	
	Preintervention baseline data were compared to postintervention data at 6, 12, 18, and 24 months. The results showed a significantly sustained reduction (P \textless .001) in total alarm rate over time. \\
	
	\textbf{Conclusion} \\
	
	A quality improvement initiative based on evidence-based practice can contribute to a sustainable reduction in nonactionable and false alarms, ultimately improving patient safety.
	
\end{quotation}

section{Definition}

`Alarm fatigue occurs when nurses are exposed to multiple alarms of mixed significance and become desensitized to alarms to the point that a critical alarm may receive no response or a delayed response.'
