\chapter{Alarm Fatigue: A Concept Analysis}

\section{Overview}

\textit{Alarm Fatigue: A Concept Analysis} (\cite{west2014alarm}) provides a very good definition of alarm fatigue in a clinical setting.

\section{Keywords}

\cite{west2014alarm} uses these keywords: concept analysis, alarm fatigue, nursing, technology, distractions

\section{Definition}

\cite{west2014alarm} quotes \cite{mccartney2012clinical} in the definition:

\begin{quote}
	Alarm fatigue, defined in the literature is the desensitization of a clinician to an alarm stimulus that results from sensory overload causing the response of an alarm to be delayed or missed \cite{mccartney2012clinical}.
\end{quote}

This definition is expanded as follows:

\begin{quotation}
	Therefore, alarm fatigue encompasses three defining attributes: 
	\begin{itemize}
		\item an environment with excessive and repeated situations;
		\item a lessened motivation and interest in surroundings;
		\item and a diminished capacity for physical and mental work.
	\end{itemize}
\end{quotation}

The cycle of alarm fatigue is described as:

\begin{quotation}
	Thus, the antecedents of alarm fatigue are:
	\begin{itemize}
		\item involvement of a healthcare professional;
		\item the ability to subjectively evaluate feelings;
		\item a patient care environment with excessive stimuli
	\end{itemize}
	The consequences of alarm fatigue are:
	\begin{itemize}
		\item a lessened capacity to give a normal response to a signal;
		\item a significant clinical event missed or ignored that could lead to a potentially harmful patient situation;
		\item limited perception of the clinical significance of the alarm signal. \\
	\end{itemize}
\end{quotation}
